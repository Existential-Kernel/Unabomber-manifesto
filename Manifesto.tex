\documentclass{article}

\title{Industrial Society and Its Future}
\author{Theodore Kaczynski}
\date{September 19, 1995}

\begin{document}
\maketitle
\section{Introduction}
The Industrial Revolution and its consequences have been a disaster for the human race.  They 
have  greatly  increased  the  life-expectancy  of  those  of  us  who  live  in  “advanced”  countries,  but  
they  have  destabilized  society,  have  made  life  unfulfilling,  have  subjected  human  beings  to  
indignities,  have  led  to  widespread  psychological  suffering  (in  the  Third  World  to  physical  
suffering  as  well)  and  have  inflicted  severe  damage  on  the  natural  world.   The  continued  
development  of  technology  will  worsen  the  situation.   It  will  certainly  subject  human  beings  to 
greater indignities and inflict greater damage on the natural world, it will probably lead to greater 
social disruption and psychological suffering, and it may lead to increased physical suffering even 
in “advanced” countries. 

The industrial-technological system may survive or it may break down.  If it survives, it may 
eventually  achieve  a  low  level  of  physical  and  psychological  suffering,  but  only  after  passing  
through a long and very painful period of adjustment and only at the cost of permanently reducing 
human beings and many other living organisms to engineered products and mere cogs in the social 
machine.   Furthermore,  if  the  system  survives,  the  consequences  will  be  inevitable:  There  is  no  
way of reforming or modifying the system so as to prevent it from depriving people of dignity and 
autonomy. If the system breaks down the consequences will still be very painful.  But the bigger the system 
grows the more disastrous the results of its breakdown will be, so if it is to break down it had best 
break down sooner rather than later. 

We therefore advocate a revolution against the industrial system.  This revolution may or may 
not make use of violence; it may be sudden or it may be a relatively gradual process spanning a 
few decades.  We can’t predict any of that.  But we do outline in a very general way the measures 
that those who hate the industrial system should take in order to prepare the way for a revolution 
against that form of society.  This is not to be a political revolution.  Its object will be to overthrow 
not governments but the economic and technological basis of the present society. 

In this article we give attention to only some of the negative developments that have grown out 
of  the  industrial-technological  system.   Other  such  developments  we  mention  only  briefly  or  
ignore altogether.  This does not mean that we regard these other developments as unimportant. 
For practical reasons we have to confine our discussion to areas that have received 
insufficient public attention or in which we have something new to say.  For example, since there 
are well-developed environmental and wilderness movements, we have written very little about 
environmental degradation or the destruction of wild nature, even though we consider these to be 
highly import

\section{The psychology of modern leftism}
Almost  everyone  will  agree  that  we  live  in  a  deeply  troubled  society.   One  of  the  most  
widespread  manifestations  of  the  craziness  of  our  world  is  leftism,  so  a  discussion  of  the psychology of leftism can serve as an introduction to the discussion of the problems of modern 
society in general.

But what is leftism? During the first half of the 20th century leftism could have been practically 
identified with socialism.  Today the movement is fragmented and it is not clear who can properly 
be  called  a  leftist. When  we  speak  of  leftists  in  this  article  we  have  in  mind  mainly  socialists, collectivists,  “politically  correct”  types,  feminists,  gay  and  disability  activists, animal  rights activists  and  the  like.   But  not  everyone  who  is  associated  with  one  of  these  movements  is  a leftist.  What we are trying to get at in discussing leftism is not so much movement or an ideology as a psychological type, or rather a collection of related types.  Thus, what we mean by “leftism” will  emerge  more  clearly  in  the  course  of  our  discussion  of  leftist  psychology.   (Also,  see paragraphs 227-230.) 

Even so, our conception of leftism will remain a good deal less clear than we would wish, but 
there doesn’t seem to be any remedy for this.  All we are trying to do here is indicate in a rough 
and approximate way the two psychological tendencies that we believe are the main driving force 
of modern leftism.  We by no means claim to be telling the whole truth about leftist 
psychology.  Also, our discussion is meant to apply to modern leftism only.  We leave open the 
question of the extent to which our discussion could be applied to the leftists of the 19th and early 
20th centuries. 

The two psychological tendencies that underlie modern leftism we call “feelings of inferiority” 
and  “oversocialization”.  Feelings  of  inferiority  are  characteristic  of  modern  leftism  as  a  whole, while  oversocialization  is  characteristic  only  of  a  certain  segment  of  modern  leftism;  but  this  segment is highly influential. 
\end{document}