\documentclass{article}
\usepackage[utf8]{inputenc}
\usepackage{afterpage}

\title{Industrial Society and Its Future}
\author{\small{Theodore John Kaczynski}}
\date{\small{September 19, 1995}}

\begin{document}
\maketitle
\small{This essay first appeared appeared in \textit{The New York Times} and \textit{The Washington Post} on Sept 19, 1995. It was published under the pseudonym FC, for \textit{Freedom Club.} The version you are viewing right now was made and formatted by Existential-Kernel on github.}
\clearpage

\section{INTRODUCTION}
\hspace{0.5cm} 1. The Industrial Revolution and its consequences have been a disaster for the human race. They have greatly increased the life-expectancy of those of us who live in “advanced” countries, but they have destabilized society, have made life unfulfilling, have subjected human beings to indignities, have led to widespread psychological suffering (in the Third World to physical suffering as well) and have inflicted severe damage on the natural world. The continued 
development of technology will worsen the situation. It will certainly subject human being to greater indignities and inflict greater damage on the natural world, it will probably lead to greater 
social disruption and psychological suffering, and it may lead to increased physical suffering even 
in “advanced” countries. \vspace{\baselineskip}

2. The industrial-technological system may survive or it may break down. If it survives, it may 
eventually achieve a low level of physical and psychological suffering, but only after passing through a long and very painful period of adjustment and only at the cost of permanently reducing human beings and many other living organisms to engineered products and mere cogs in the social 
machine. Furthermore, if the system survives, the consequences will be inevitable: There is no way of reforming or modifying the system so as to prevent it from depriving people of dignity and 
autonomy. \vspace{\baselineskip}

3. If the system breaks down the consequences will still be very painful. But the bigger the system 
grows the more disastrous the results of its breakdown will be, so if it is to break down it had best 
break down sooner rather than later. \vspace{\baselineskip}

4. We therefore advocate a revolution against the industrial system. This revolution may or may 
not make use of violence; it may be sudden or it may be a relatively gradual process spanning a 
few decades. We can’t predict any of that. But we do outline in a very general way the measures 
that those who hate the industrial system should take in order to prepare the way for a revolution 
against that form of society. This is not to be a political revolution. Its object will be to overthrow not governments but the economic and technological basis of the present society. \vspace{\baselineskip}

5. In this article we give attention to only some of the negative developments that have grown out 
of the industrial-technological system. Other such developments we mention only briefly or 
ignore altogether. This does not mean that we regard these other developments as 
unimportant. For practical reasons we have to confine our discussion to areas that have received 
insufficient public attention or in which we have something new to say. For example, since there 
are well-developed environmental and wilderness movements, we have written very little about 
environmental degradation or the destruction of wild nature, even though we consider these to be 
highly important.


\section{THE PSYCHOLOGY OF MODERN LEFTISM}

\hspace{0.5cm} 6. Almost everyone will agree that we live in a deeply troubled society. One of the most 
widespread manifestations of the craziness of our world is leftism, so a discussion of the 
psychology of leftism can serve as an introduction to the discussion of the problems of modern 
society in general. \vspace{\baselineskip}

7. But what is leftism? During the first half of the 20th century leftism could have been practically 
identified with socialism. Today the movement is fragmented and it is not clear who can properly 
be called a leftist. When we speak of leftists in this article we have in mind mainly socialists, 
collectivists, “politically correct” types, feminists, gay and disability activists, animal rights 
activists and the like. But not everyone who is associated with one of these movements is a 
leftist. What we are trying to get at in discussing leftism is not so much movement or an ideology 
as a psychological type, or rather a collection of related types. Thus, what we mean by “leftism” 
will emerge more clearly in the course of our discussion of leftist psychology. (Also, see paragraphs 227-230.) \vspace{\baselineskip}

8. Even so, our conception of leftism will remain a good deal less clear than we would wish, but 
there doesn’t seem to be any remedy for this. All we are trying to do here is indicate in a rough 
and approximate way the two psychological tendencies that we believe are the main driving force 
of modern leftism. We by no means claim to be telling the whole truth about leftist 
psychology. Also, our discussion is meant to apply to modern leftism only. We leave open the 
question of the extent to which our discussion could be applied to the leftists of the 19th and early 
20th centuries. \vspace{\baselineskip}

9. The two psychological tendencies that underlie modern leftism we call “feelings of inferiority” 
and “oversocialization”. Feelings of inferiority are characteristic of modern leftism as a whole, 
while oversocialization is characteristic only of a certain segment of modern leftism; but this 
segment is highly influential.

\section{FEELINGS OF INFERIORITY}


\hspace{0.5cm} 10. By “feelings of inferiority” we mean not only inferiority feelings in the strict sense but a whole 
spectrum of related traits; low self-esteem, feelings of powerlessness, depressive tendencies, 
defeatism, guilt, self-hatred, etc. We argue that modern leftists tend to have some such feelings 
(possibly more or less repressed) and that these feelings are decisive in determining the direction 
of modern leftism. \vspace{\baselineskip}

11. When someone interprets as derogatory almost anything that is said about him (or about 
groups with whom he identifies) we conclude that he has inferiority feelings or low self-esteem. This tendency is pronounced among minority rights activists, whether or not they belong 
to the minority groups whose rights they defend. They are hypersensitive about the words used to 
designate minorities and about anything that is said concerning minorities. The terms “negro”, 
“oriental”, “handicapped” or “chick” for an African, an Asian, a disabled person or a woman 
originally had no derogatory connotation. “Broad” and “chick” were merely the feminine 
equivalents of “guy”, “dude” or “fellow”. The negative connotations have been attached to these 
terms by the activists themselves. Some animal rights activists have gone so far as to reject the 
word “pet” and insist on its replacement by “animal companion”. Leftish anthropologists go to 
great lengths to avoid saying anything about primitive peoples that could conceivably be 
interpreted as negative. They want to replace the word “primitive” by “nonliterate”. They seem 
almost paranoid about anything that might suggest that any primitive culture is inferior to our 
own. (We do not mean to imply that primitive cultures are inferior to ours. We merely point out 
the hyper sensitivity of leftish anthropologists.) \vspace{\baselineskip}

12. Those who are most sensitive about “politically incorrect” terminology are not the average 
black ghetto-dweller, Asian immigrant, abused woman or disabled person, but a minority of 
activists, many of whom do not even belong to any “oppressed” group but come from privileged 
strata of society. Political correctness has its stronghold among university professors, who have 
secure employment with comfortable salaries, and the majority of whom are heterosexual white 
males from middle- to upper-middle-class families. \vspace{\baselineskip}

13. Many leftists have an intense identification with the problems of groups that have an image 
of being weak (women), defeated (American Indians), repellent (homosexuals) or otherwise 
inferior. The leftists themselves feel that these groups are inferior. They would never admit to 
themselves that they have such feelings, but it is precisely because they do see these groups as 
inferior that they identify with their problems. (We do not mean to suggest that women, Indians, 
etc. are inferior; we are only making a point about leftist psychology.) \vspace{\baselineskip}

14. Feminists are desperately anxious to prove that women are as strong and as capable as 
men. Clearly they are nagged by a fear that women may not be as strong and as capable as men. \vspace{\baselineskip}

15. Leftists tend to hate anything that has an image of being strong, good and successful. They 
hate America, they hate Western civilization, they hate white males, they hate rationality. The 
reasons that leftists give for hating the West, etc. clearly do not correspond with their real 
motives. They say they hate the West because it is warlike, imperialistic, sexist, ethnocentric and 
so forth, but where these same faults appear in socialist countries or in primitive cultures, the leftist 
finds excuses for them, or at best he grudgingly admits that they exist; whereas he enthusiastically 
points out (and often greatly exaggerates) these faults where they appear in Western 
civilization. Thus it is clear that these faults are not the leftist’s real motive for hating America 
and the West. He hates America and the West because they are strong and successful. \vspace{\baselineskip}

16. Words like “self-confidence”, “self-reliance”, “initiative”, “enterprise”, “optimism”, etc., play 
little role in the liberal and leftist vocabulary. The leftist is anti-individualistic, pro-
collectivist. He wants society to solve every one’s problems for them, satisfy everyone’s needs 
for them, take care of them. He is not the sort of person who has an inner sense of confidence in 
his ability to solve his own problems and satisfy his own needs. The leftist is antagonistic to the 
concept of competition because, deep inside, he feels like a loser. \vspace{\baselineskip}

17. Art forms that appeal to modern leftish intellectuals tend to focus on sordidness, defeat and 
despair, or else they take an orgiastic tone, throwing off rational control as if there were no hope 
of accomplishing anything through rational calculation and all that was left was to immerse oneself 
in the sensations of the moment. \vspace{\baselineskip}

18. Modern leftish philosophers tend to dismiss reason, science, and objective reality, and to insist 
that everything is culturally relative.  It is true that one can ask serious questions about the 
foundations of scientific knowledge and about how, if at all, the concept of objective reality can 
be defined. But it is obvious that modern leftish philosophers are not simply cool-headed logicians 
systematically analyzing the foundations of knowledge. They are deeply involved emotionally in 
their attack on truth and reality. They attack these concepts because of their own psychological 
needs. For one thing, their attack is an outlet for hostility, and, to the extent that it is successful, it 
satisfies the drive for power. More importantly, the leftist hates science and rationality because 
they classify certain beliefs as true (i.e., successful, superior) and other beliefs as false (i.e., failed, 
inferior). The leftist’s feelings of inferiority run so deep that he cannot tolerate any classification 
of some things as successful or superior and other things as failed or inferior. This also underlies 
the rejection by many leftists of the concept of mental illness and of the utility of IQ tests. Leftists 
are antagonistic to genetic explanations of human abilities or behavior because such explanations 
tend to make some persons appear superior or inferior to others. Leftists prefer to give society the 
credit or blame for an individual’s ability or lack of it. Thus if a person is “inferior” it is not his 
fault, but society’s, because he has not been brought up properly. \vspace{\baselineskip}

19. The leftist is not typically the kind of person whose feelings of inferiority make him a braggart,
an egotist, a bully, a self-promoter, a ruthless competitor. This kind of person has not wholly lost 
faith in himself. He has a deficit in his sense of power and self-worth, but he can still conceive of 
himself as having the capacity to be strong, and his efforts to make himself strong produce his 
unpleasant behavior.\textsuperscript{1} But the leftist is too far gone for that. His feelings of inferiority are so 
ingrained that he cannot conceive of himself as individually strong and valuable. Hence the 
collectivism of the leftist. He can feel strong only as a member of a large organization or a mass 
movement with which he identifies himself. \vspace{\baselineskip}

20. Notice the masochistic tendency of leftist tactics. Leftists protest by lying down in front of 
vehicles, they intentionally provoke police or racists to abuse them, etc. These tactics may often 
be effective, but many leftists use them not as a means to an end but because they prefer 
masochistic tactics. Self-hatred is a leftist trait. \vspace{\baselineskip}

21. Leftists may claim that their activism is motivated by compassion or by moral principles, and 
moral principle does play a role for the leftist of the oversocialized type. But compassion and 
moral principle cannot be the main motives for leftist activism. Hostility is too prominent a 
component of leftist behavior; so is the drive for power. Moreover, much leftist behavior is not 
rationally calculated to be of benefit to the people whom the leftists claim to be trying to help. For 
example, if one believes that affirmative action is good for black people, does it make sense to 
demand affirmative action in hostile or dogmatic terms? Obviously it would be more productive 
to take a diplomatic and conciliatory approach that would make at least verbal and symbolic 
concessions to white people who think that affirmative action discriminates against them. But 
leftist activists do not take such an approach because it would not satisfy their emotional 
needs. Helping black people is not their real goal. Instead, race problems serve as an excuse for 
them to express their own hostility and frustrated need for power. In doing so they actually harm 
black people, because the activists’ hostile attitude toward the white majority tends to intensify 
race hatred. \vspace{\baselineskip}

22. If our society had no social problems at all, the leftists would have to invent problems in order 
to provide themselves with an excuse for making a fuss. \vspace{\baselineskip} \newpage

23. We emphasize that the foregoing does not pretend to be an accurate description of everyone 
who might be considered a leftist. It is only a rough indication of a general tendency of leftism. \vspace{\baselineskip}


\section{OVERSOCIALIZATION}

\hspace{0.5cm} 24. Psychologists use the term “socialization” to designate the process by which children are 
trained to think and act as society demands. A person is said to be well socialized if he believes 
in and obeys the moral code of his society and fits in well as a functioning part of that society. It 
may seem senseless to say that many leftists are over-socialized, since the leftist is perceived as a 
rebel. Nevertheless, the position can be defended. Many leftists are not such rebels as they seem. \vspace{\baselineskip}

25. The moral code of our society is so demanding that no one can think, feel and act in a 
completely moral way. For example, we are not supposed to hate anyone, yet almost everyone 
hates somebody at some time or other, whether he admits it to himself or not. Some people are so 
highly socialized that the attempt to think, feel and act morally imposes a severe burden on 
them. In order to avoid feelings of guilt, they continually have to deceive themselves about their 
own motives and find moral explanations for feelings and actions that in reality have a non-moral 
origin. We use the term “oversocialized” to describe such people.\textsuperscript{2} \vspace{\baselineskip}

26. Oversocialization can lead to low self-esteem, a sense of powerlessness, defeatism, guilt, 
etc. One of the most important means by which our society socializes children is by making them 
feel ashamed of behavior or speech that is contrary to society’s expectations. If this is overdone, 
or if a particular child is especially susceptible to such feelings, he ends by feeling ashamed of 
himself. Moreover the thought and the behavior of the oversocialized person are more restricted 
by society’s expectations than are those of the lightly socialized person. The majority of people 
engage in a significant amount of naughty behavior. They lie, they commit petty thefts, they break 
traffic laws, they goof off at work, they hate someone, they say spiteful things or they use some 
underhanded trick to get ahead of the other guy. The oversocialized person cannot do these things, 
or if he does do them he generates in himself a sense of shame and self-hatred. The oversocialized 
person cannot even experience, without guilt, thoughts or feelings that are contrary to the accepted 
morality; he cannot think “unclean” thoughts. And socialization is not just a matter of morality; 
we are socialized to conform to many norms of behavior that do not fall under the heading of 
morality. Thus the oversocialized person is kept on a psychological leash and spends his life 
running on rails that society has laid down for him. In many oversocialized people this results in 
a sense of constraint and powerlessness that can be a severe hardship. We suggest that 
oversocialization is among the more serious cruelties that human being inflict on one another. \vspace{\baselineskip} \newpage

27. We argue that a very important and influential segment of the modern left is oversocialized 
and that their oversocialization is of great importance in determining the direction of modern 
leftism. Leftists of the oversocialized type tend to be intellectuals or members of the upper-middle 
class. Notice that university intellectuals\textsuperscript{3} constitute the most highly socialized segment of our society and also the most leftwing segment. \vspace{\baselineskip}

28. The leftist of the oversocialized type tries to get off his psychological leash and assert his 
autonomy by rebelling. But usually he is not strong enough to rebel against the most basic values 
of society. Generally speaking, the goals of today’s leftists are not in conflict with the accepted 
morality. On the contrary, the left takes an accepted moral principle, adopts it as its own, and then 
accuses mainstream society of violating that principle. Examples: racial equality, equality of the 
sexes, helping poor people, peace as opposed to war, nonviolence generally, freedom of 
expression, kindness to animals. More fundamentally, the duty of the individual to serve society 
and the duty of society to take care of the individual. All these have been deeply rooted values of 
our society (or at least of its middle and upper classes\textsuperscript{4} for a long time. These values are explicitly 
or implicitly expressed or presupposed in most of the material presented to us by the mainstream 
communications media and the educational system. Leftists, especially those of the oversocialized 
type, usually do not rebel against these principles but justify their hostility to society by claiming 
(with some degree of truth) that society is not living up to these principles. \vspace{\baselineskip}

29. Here is an illustration of the way in which the oversocialized leftist shows his real attachment 
to the conventional attitudes of our society while pretending to be in rebellion aginst it. Many 
leftists push for affirmative action, for moving black people into high-prestige jobs, for improved 
education in black schools and more money for such schools; the way of life of the black 
“underclass” they regard as a social disgrace. They want to integrate the black man into the 
system, make him a business executive, a lawyer, a scientist just like upper-middle-class white 
people. The leftists will reply that the last thing they want is to make the black man into a copy of 
the white man; instead, they want to preserve African American culture. But in what does this 
preservation of African American culture consist? It can hardly consist in anything more than 
eating black-style food, listening to black-style music, wearing black-style clothing and going to a 
black-style church or mosque. In other words, it can express itself only in superficial matters. In 
all essential respects most leftists of the oversocialized type want to make the black man conform 
to white, middle-class ideals. They want to make him study technical subjects, become an 
executive or a scientist, spend his life climbing the status ladder to prove that black people are as 
good as white. They want to make black fathers “responsible,” they want black gangs to become 
nonviolent, etc. But these are exactly the values of the industrial-technological system. The 
system couldn’t care less what kind of music a man listens to, what kind of clothes he wears or 
what religion he believes in as long as he studies in school, holds a respectable job, climbs the 
status ladder, is a “responsible” parent, is nonviolent and so forth. In effect, however much he 
may deny it, the oversocialized leftist wants to integrate the black man into the system and make 
him adopt its values. \vspace{\baselineskip}

30. We certainly do not claim that leftists, even of the oversocialized type, never rebel against the 
fundamental values of our society. Clearly they sometimes do. Some oversocialized leftists have 
gone so far as to rebel against one of modern society’s most important principles by engaging in 
physical violence. By their own account, violence is for them a form of “liberation.” In other 
words, by committing violence they break through the psychological restraints that have been 
trained into them. Because they are oversocialized these restraints have been more confining for 
them than for others; hence their need to break free of them. But they usually justify their rebellion 
in terms of mainstream values. If they engage in violence they claim to be fighting against racism 
or the like. \vspace{\baselineskip}

31. We realize that many objections could be raised to the foregoing thumbnail sketch of leftist 
psychology. The real situation is complex, and anything like a complete description of it would 
take several volumes even if the necessary data were available. We claim only to have indicated 
very roughly the two most important tendencies in the psychology of modern leftism. \vspace{\baselineskip}

32. The problems of the leftist are indicative of the problems of our society as a whole. Low self-
esteem, depressive tendencies and defeatism are not restricted to the left. Though they are 
especially noticeable in the left, they are widespread in our society. And today’s society tries to 
socialize us to a greater extent than any previous society. We are even told by experts how to eat, 
how to exercise, how to make love, how to raise our kids and so forth. 


\section{THE POWER PROCESS}

\hspace{0.5cm} 33. Human beings have a need (probably based in biology) for something that we will call 
the power process. This is closely related to the need for power (which is widely recognized) but 
is not quite the same thing. The power process has four elements. The three most clear-cut of 
these we call goal, effort and attainment of goal. (Everyone needs to have goals whose attainment 
requires effort, and needs to succeed in attaining at least some of his goals.) The fourth element is 
more difficult to define and may not be necessary for everyone. We call it autonomy and will 
discuss it later (paragraphs 42-44). \vspace{\baselineskip}

34. Consider the hypothetical case of a man who can have anything he wants just by wishing for 
it. Such a man has power, but he will develop serious psychological problems. At first he will 
have a lot of fun, but by and by he will become acutely bored and demoralized. Eventually he may 
become clinically depressed. History shows that leisured aristocracies tend to become 
decadent. This is not true of fighting aristocracies that have to struggle to maintain their 
power. But leisured, secure aristocracies that have no need to exert themselves usually become 
bored, hedonistic and demoralized, even though they have power. This shows that power is not 
enough. One must have goals toward which to exercise one’s power. \vspace{\baselineskip}

35. Everyone has goals; if nothing else, to obtain the physical necessities of life: food, water and 
whatever clothing and shelter are made necessary by the climate. But the leisured aristocrat 
obtains these things without effort. Hence his boredom and demoralization. \vspace{\baselineskip} \newpage

36. Nonattainment of important goals results in death if the goals are physical necessities, and in 
frustration if non-attainment of the goals is compatible with survival. Consistent failure to attain 
goals throughout life results in defeatism, low self-esteem or depression. \vspace{\baselineskip}

37. Thus, in order to avoid serious psychological problems, a human being needs goals whose 
attainment requires effort, and he must have a reasonable rate of success in attaining his goals. 


\section{SURROGATE ACTIVITIES}

\hspace{0.5cm} 38. But not every leisured aristocrat becomes bored and demoralized. For example, the emperor 
Hirohito, instead of sinking into decadent hedonism, devoted himself to marine biology, a field in 
which he became distinguished. When people do not have to exert themselves to satisfy their 
physical needs they often set up artificial goals for themselves. In many cases they then pursue 
these goals with the same energy and emotional involvement that they otherwise would have put 
into the search for physical necessities. Thus the aristocrats of the Roman Empire had their literary 
pretensions; many European aristocrats a few centuries ago invested tremendous time and energy 
in hunting, though they certainly didn’t need the meat; other aristocracies have competed for status 
through elaborate displays of wealth; and a few aristocrats, like Hirohito, have turned to science. \vspace{\baselineskip}

39. We use the term “surrogate activity” to designate an activity that is directed toward an artificial 
goal that people set up for themselves merely in order to have some goal to work toward, or let us 
say, merely for the sake of the “fulfillment” that they get from pursuing the goal. Here is a rule of 
thumb for the identification of surrogate activities. Given a person who devotes much time and 
energy to the pursuit of goal X, ask yourself this: If he had to devote most of his time and energy 
to satisfying his biological needs, and if that effort required him to use his physical and mental 
faculties in a varied and interesting way, would he feel seriously deprived because he did not attain 
goal X? If the answer is no, then the person’s pursuit of goal X is a surrogate activity. Hirohito’s 
studies in marine biology clearly constituted a surrogate activity, since it is pretty certain that if 
Hirohito had had to spend his time working at interesting non-scientific tasks in order to obtain the 
necessities of life, he would not have felt deprived because he didn’t know all about the anatomy 
and life-cycles of marine animals. On the other hand the pursuit of sex and love (for example) is 
not a surrogate activity, because most people, even if their existence were otherwise satisfactory, 
would feel deprived if they passed their lives without ever having a relationship with a member of 
the opposite sex. (But pursuit of an excessive amount of sex, more than one really needs, can be 
a surrogate activity.) \vspace{\baselineskip}

40. In modern industrial society only minimal effort is necessary to satisfy one’s physical 
needs. It is enough to go through a training program to acquire some petty technical skill, then 
come to work on time and exert the very modest effort needed to hold a job. The only requirements 
are a moderate amount of intelligence and, most of all, simple obedience. If one has those, society 
takes care of one from cradle to grave. (Yes, there is an underclass that cannot take the physical 
necessities for granted, but we are speaking here of mainstream society.) Thus it is not surprising 
that modern society is full of surrogate activities. These include scientific work, athletic 
achievement, humanitarian work, artistic and literary creation, climbing the corporate ladder, 
acquisition of money and material goods far beyond the point at which they cease to give any 
additional physical satisfaction, and social activism when it addresses issues that are not important 
for the activist personally, as in the case of white activists who work for the rights of nonwhite 
minorities. These are not always pure surrogate activities, since for many people they may be 
motivated in part by needs other than the need to have some goal to pursue. Scientific work may 
be motivated in part by a drive for prestige, artistic creation by a need to express feelings, militant 
social activism by hostility. But for most people who pursue them, these activities are in large part 
surrogate activities. For example, the majority of scientists will probably agree that the 
“fulfillment” they get from their work is more important than the money and prestige they earn. \vspace{\baselineskip}

41. For many if not most people, surrogate activities are less satisfying than the pursuit of real 
goals (that is, goals that people would want to attain even if their need for the power process were 
already fulfilled). One indication of this is the fact that, in many or most cases, people who are 
deeply involved in surrogate activities are never satisfied, never at rest. Thus the money-maker 
constantly strives for more and more wealth. The scientist no sooner solves one problem than he 
moves on to the next. The long-distance runner drives himself to run always farther and 
faster. Many people who pursue surrogate activities will say that they get far more fulfillment 
from these activities than they do from the “mundane” business of satisfying their biological needs, 
but that is because in our society the effort needed to satisfy the biological needs has been reduced 
to triviality. More importantly, in our society people do not satisfy their biological needs 
autonomously but by functioning as parts of an immense social machine. In contrast, people 
generally have a great deal of autonomy in pursuing their surrogate activities. 


\section{AUTONOMY}

\hspace{0.5cm} 42. Autonomy as a part of the power process may not be necessary for every individual. But most 
people need a greater or lesser degree of autonomy in working toward their goals. Their efforts 
must be undertaken on their own initiative and must be under their own direction and control. Yet 
most people do not have to exert this initiative, direction and control as single individuals. It is 
usually enough to act as a member of a small group. Thus if half a dozen people discuss a goal 
among themselves and make a successful joint effort to attain that goal, their need for the power 
process will be served. But if they work under rigid orders handed down from above that leave 
them no room for autonomous decision and initiative, then their need for the power process will 
not be served. The same is true when decisions are made on a collective basis if the group making 
the collective decision is so large that the role of each individual is insignificant.\textsuperscript{5} \vspace{\baselineskip} \newpage

43. It is true that some individuals seem to have little need for autonomy. Either their drive for 
power is weak or they satisfy it by identifying themselves with some powerful organization to 
which they belong. And then there are unthinking, animal types who seem to be satisfied with a 
purely physical sense of power (the good combat soldier, who gets his sense of power by 
developing fighting skills that he is quite content to use in blind obedience to his superiors). \vspace{\baselineskip}

44. But for most people it is through the power process—having a goal, making an autonomous 
effort and attaining the goal—that self-esteem, self-confidence and a sense of power are 
acquired. When one does not have adequate opportunity to go through the power process the 
consequences are (depending on the individual and on the way the power process is disrupted) 
boredom, demoralization, low self-esteem, inferiority feelings, defeatism, depression, anxiety, 
guilt, frustration, hostility, spouse or child abuse, insatiable hedonism, abnormal sexual behavior, 
sleep disorders, eating disorders, etc.\textsuperscript{6}


\section{SOURCES OF SOCIAL PROBLEMS}

\hspace{0.5cm} 45. Any of the foregoing symptoms can occur in any society, but in modern industrial society they 
are present on a massive scale. We aren’t the first to mention that the world today seems to be 
going crazy. This sort of thing is not normal for human societies. There is good reason to believe 
that primitive man suffered from less stress and frustration and was better satisfied with his way 
of life than modern man is. It is true that not all was sweetness and light in primitive 
societies. Abuse of women was common among the Australian aborigines, transexuality was fairly 
common among some of the American Indian tribes. But it does appear that generally speaking 
the kinds of problems that we have listed in the preceding paragraph were far less common among 
primitive peoples than they are in modern society. \vspace{\baselineskip}

46. We attribute the social and psychological problems of modern society to the fact that that 
society requires people to live under conditions radically different from those under which the 
human race evolved and to behave in ways that conflict with the patterns of behavior that the 
human race developed while living under the earlier conditions. It is clear from what we have 
already written that we consider lack of opportunity to properly experience the power process as 
the most important of the abnormal conditions to which modern society subjects people. But it is 
not the only one. Before dealing with disruption of the power process as a source of social 
problems we will discuss some of the other sources. \vspace{\baselineskip}

47. Among the abnormal conditions present in modern industrial society are excessive density of 
population, isolation of man from nature, excessive rapidity of social change and the breakdown 
of natural small-scale communities such as the extended family, the village or the tribe. \vspace{\baselineskip}

48. It is well known that crowding increases stress and aggression. The degree of crowding that 
exists today and the isolation of man from nature are consequences of technological progress. All 
pre-industrial societies were predominantly rural. The Industrial Revolution vastly increased the 
size of cities and the proportion of the population that lives in them, and modern agricultural 
technology has made it possible for the Earth to support a far denser population than it ever did 
before. (Also, technology exacerbates the effects of crowding because it puts increased disruptive 
powers in people’s hands. For example, a variety of noise-making devices: power mowers, radios, 
motorcycles, etc. If the use of these devices is unrestricted, people who want peace and quiet are 
frustrated by the noise. If their use is restricted, people who use the devices are frustrated by the 
regulations. But if these machines had never been invented there would have been no conflict and 
no frustration generated by them.) \vspace{\baselineskip}

49. For primitive societies the natural world (which usually changes only slowly) provided a 
stable framework and therefore a sense of security. In the modern world it is human society that 
dominates nature rather than the other way around, and modern society changes very rapidly owing 
to technological change. Thus there is no stable framework. \vspace{\baselineskip}
 
50. The conservatives are fools: They whine about the decay of traditional values, yet they 
enthusiastically support technological progress and economic growth. Apparently it never occurs 
to them that you can’t make rapid, drastic changes in the technology and the economy of a society 
without causing rapid changes in all other aspects of the society as well, and that such rapid 
changes inevitably break down traditional values. \vspace{\baselineskip}

51. The breakdown of traditional values to some extent implies the breakdown of the bonds that 
hold together traditional small-scale social groups. The disintegration of small-scale social groups 
is also promoted by the fact that modern conditions often require or tempt individuals to move to 
new locations, separating themselves from their communities. Beyond that, a technological 
society has to weaken family ties and local communities if it is to function efficiently. In modern 
society an individual’s loyalty must be first to the system and only secondarily to a small-scale 
community, because if the internal loyalties of small-scale communities were stronger than loyalty 
to the system, such communities would pursue their own advantage at the expense of the system. \vspace{\baselineskip}

52. Suppose that a public official or a corporation executive appoints his cousin, his friend or his 
co-religionist to a position rather than appointing the person best qualified for the job. He has 
permitted personal loyalty to supersede his loyalty to the system, and that is “nepotism” or 
“discrimination,” both of which are terrible sins in modern society. Would-be industrial societies 
that have done a poor job of subordinating personal or local loyalties to loyalty to the system are 
usually very inefficient. (Look at Latin America.) Thus an advanced industrial society can tolerate 
only those small-scale communities that are emasculated, tamed and made into tools of the 
system.\textsuperscript{7} \vspace{\baselineskip}

53. Crowding, rapid change and the breakdown of communities have been widely recognized as 
sources of social problems. But we do not believe they are enough to account for the extent of the 
problems that are seen today. \vspace{\baselineskip} \newpage

54. A few pre-industrial cities were very large and crowded, yet their inhabitants do not seem to 
have suffered from psychological problems to the same extent as modern man. In America today 
there still are uncrowded rural areas, and we find there the same problems as in urban areas, though 
the problems tend to be less acute in the rural areas. Thus crowding does not seem to be the 
decisive factor. \vspace{\baselineskip}

55. On the growing edge of the American frontier during the 19th century, the mobility of the 
population probably broke down extended families and small-scale social groups to at least the 
same extent as these are broken down today. In fact, many nuclear families lived by choice in 
such isolation, having no neighbors within several miles, that they belonged to no community at 
all, yet they do not seem to have developed problems as a result. \vspace{\baselineskip}

56. Furthermore, change in American frontier society was very rapid and deep. A man might be 
born and raised in a log cabin, outside the reach of law and order and fed largely on wild meat; 
and by the time he arrived at old age he might be working at a regular job and living in an ordered 
community with effective law enforcement. This was a deeper change than that which typically 
occurs in the life of a modern individual, yet it does not seem to have led to psychological 
problems. In fact, 19th century American society had an optimistic and self-confident tone, quite 
unlike that of today’s society.\textsuperscript{8} \vspace{\baselineskip}

57. The difference, we argue, is that modern man has the sense (largely justified) that change is 
imposed on him, whereas the 19th century frontiersman had the sense (also largely justified) that 
he created change himself, by his own choice. Thus a pioneer settled on a piece of land of his own 
choosing and made it into a farm through his own effort. In those days an entire county might 
have only a couple of hundred inhabitants and was a far more isolated and autonomous entity than 
a modern county is. Hence the pioneer farmer participated as a member of a relatively small group 
in the creation of a new, ordered community. One may well question whether the creation of this 
community was an improvement, but at any rate it satisfied the pioneer’s need for the power 
process. \vspace{\baselineskip}

58. It would be possible to give other examples of societies in which there has been rapid change 
and/or lack of close community ties without the kind of massive behavioral aberration that is seen 
in today’s industrial society. We contend that the most important cause of social and psychological 
problems in modern society is the fact that people have insufficient opportunity to go through the 
power process in a normal way. We don’t mean to say that modern society is the only one in 
which the power process has been disrupted. Probably most if not all civilized societies have 
interfered with the power process to a greater or lesser extent. But in modern industrial society 
the problem has become particularly acute. Leftism, at least in its recent (mid- to late-20th century) 
form, is in part a symptom of deprivation with respect to the power process. 


\section{DISRUPTION OF THE POWER PROCESS IN MODERN SOCIETY}

\hspace{0.5cm} 59. We divide human drives into three groups: (1) those drives that can be satisfied with minimal 
effort; (2) those that can be satisfied but only at the cost of serious effort; (3) those that cannot be 
adequately satisfied no matter how much effort one makes. The power process is the process of 
satisfying the drives of the second group. The more drives there are in the third group, the more 
there is frustration, anger, eventually defeatism, depression, etc. \vspace{\baselineskip}

60. In modern industrial society natural human drives tend to be pushed into the first and third 
groups, and the second group tends to consist increasingly of artificially created drives. \vspace{\baselineskip}

61. In primitive societies, physical necessities generally fall into group 2: They can be obtained, 
but only at the cost of serious effort. But modern society tends to guaranty the physical necessities 
to everyone\textsuperscript{9} in exchange for only minimal effort, hence physical needs are pushed into group 1. 
(There may be disagreement about whether the effort needed to hold a job is “minimal”; but 
usually, in lower- to middle-level jobs, whatever effort is required is merely that of 
obedience. You sit or stand where you are told to sit or stand and do what you are told to do in the 
way you are told to do it. Seldom do you have to exert yourself seriously, and in any case you 
have hardly any autonomy in work, so that the need for the power process is not well served.) \vspace{\baselineskip}

62. Social needs, such as sex, love and status, often remain in group 2 in modern society, 
depending on the situation of the individual.\textsuperscript{10} But, except for people who have a particularly 
strong drive for status, the effort required to fulfill the social drives is insufficient to satisfy 
adequately the need for the power process. \vspace{\baselineskip}

63. So certain artificial needs have been created that fall into group 2, hence serve the need for the 
power process. Advertising and marketing techniques have been developed that make many 
people feel they need things that their grandparents never desired or even dreamed of. It requires 
serious effort to earn enough money to satisfy these artificial needs, hence they fall into group 2. 
(But see paragraphs 80-82.) Modern man must satisfy his need for the power process largely 
through pursuit of the artificial needs created by the advertising and marketing industry\textsuperscript{11}, and 
through surrogate activities. \vspace{\baselineskip}

64. It seems that for many people, maybe the majority, these artificial forms of the power process 
are insufficient. A theme that appears repeatedly in the writings of the social critics of the second 
half of the 20th century is the sense of purposelessness that afflicts many people in modern 
society. (This purposelessness is often called by other names such as “anomic” or “middle-class 
vacuity.”) We suggest that the so-called “identity crisis” is actually a search for a sense of purpose, 
often for commitment to a suitable surrogate activity. It may be that existentialism is in large part 
a response to the purposelessness of modern life\textsuperscript{12}. Very widespread in modern society is the search 
for “fulfillment.” But we think that for the majority of people an activity whose main goal is 
fulfillment (that is, a surrogate activity) does not bring completely satisfactory fulfillment. In other 
words, it does not fully satisfy the need for the power process. (See paragraph 41.) That need can 
be fully satisfied only through activities that have some external goal, such as physical necessities, 
sex, love, status, revenge, etc. \vspace{\baselineskip}

65. Moreover, where goals are pursued through earning money, climbing the status ladder or 
functioning as part of the system in some other way, most people are not in a position to pursue 
their goals autonomously. Most workers are someone else’s employee and, as we pointed out in 
paragraph 61, must spend their days doing what they are told to do in the way they are told to do 
it. Even most people who are in business for themselves have only limited autonomy. It is a 
chronic complaint of small-business persons and entrepreneurs that their hands are tied by 
excessive government regulation. Some of these regulations are doubtless unnecessary, but for 
the most part government regulations are essential and inevitable parts of our extremely complex 
society. A large portion of small business today operates on the franchise system. It was reported 
in the Wall Street Journal a few years ago that many of the franchise-granting companies require 
applicants for franchises to take a personality test that is designed to exclude those who have 
creativity and initiative, because such persons are not sufficiently docile to go along obediently 
with the franchise system. This excludes from small business many of the people who most need 
autonomy. \vspace{\baselineskip}

66. Today people live more by virtue of what the system does for them or to them than by virtue 
of what they do for themselves. And what they do for themselves is done more and more along 
channels laid down by the system. Opportunities tend to be those that the system provides, the 
opportunities must be exploited in accord with rules and regulations,\textsuperscript{13} and techniques prescribed 
by experts must be followed if there is to be a chance of success. \vspace{\baselineskip}

67. Thus the power process is disrupted in our society through a deficiency of real goals and a 
deficiency of autonomy in the pursuit of goals. But it is also disrupted because of those human 
drives that fall into group 3: the drives that one cannot adequately satisfy no matter how much 
effort one makes. One of these drives is the need for security. Our lives depend on decisions made 
by other people; we have no control over these decisions and usually we do not even know the 
people who make them. (“We live in a world in which relatively few people maybe 500 or 
1,000 make the important decisions”, Philip B. Heymann of Harvard Law School, quoted by 
Anthony Lewis, New York Times, April 21, 1995.) Our lives depend on whether safety standards 
at a nuclear power plant are properly maintained; on how much pesticide is allowed to get into our 
food or how much pollution into our air; on how skillful (or incompetent) our doctor is; whether 
we lose or get a job may depend on decisions made by government economists or corporation 
executives; and so forth. Most individuals are not in a position to secure themselves against these 
threats to more than a very limited extent. The individual’s search for security is therefore 
frustrated, which leads to a sense of powerlessness. \vspace{\baselineskip}

68. It may be objected that primitive man is physically less secure than modern man, as is shown 
by his shorter life expectancy; hence modern man suffers from less, not more than the amount of 
insecurity that is normal for human beings. But psychological security does not closely correspond 
with physical security. What makes us feel secure is not so much objective security as a sense of 
confidence in our ability to take care of ourselves. Primitive man, threatened by a fierce animal 
or by hunger, can fight in self-defense or travel in search of food. He has no certainty of success 
in these efforts, but he is by no means helpless against the things that threaten him. The modern 
individual on the other hand is threatened by many things against which he is helpless: nuclear 
accidents, carcinogens in food, environmental pollution, war, increasing taxes, invasion of his 
privacy by large organizations, nationwide social or economic phenomena that may disrupt his 
way of life. \vspace{\baselineskip}

69. It is true that primitive man is powerless against some of the things that threaten him; disease 
for example. But he can accept the risk of disease stoically. It is part of the nature of things, it is 
no one’s fault, unless it is the fault of some imaginary, impersonal demon. But threats to the 
modern individual tend to be man-made. They are not the results of chance but are imposed on 
him by other persons whose decisions he, as an individual, is unable to influence. Consequently 
he feels frustrated, humiliated and angry. \vspace{\baselineskip}

70. Thus primitive man for the most part has his security in his own hands (either as an individual 
or as a member of a small group) whereas the security of modern man is in the hands of persons 
or organizations that are too remote or too large for him to be able personally to influence them. So 
modern man’s drive for security tends to fall into groups 1 and 3; in some areas (food, shelter etc.) 
his security is assured at the cost of only trivial effort, whereas in other areas he cannot attain 
security. (The foregoing greatly simplifies the real situation, but it does indicate in a rough, general 
way how the condition of modern man differs from that of primitive man.) \vspace{\baselineskip}

71. People have many transitory drives or impulses that are necessarily frustrated in modern life, 
hence fall into group 3. One may become angry, but modern society cannot permit fighting. In 
many situations it does not even permit verbal aggression. When going somewhere one may be in 
a hurry, or one may be in a mood to travel slowly, but one generally has no choice but to move 
with the flow of traffic and obey the traffic signals. One may want to do one’s work in a different 
way, but usually one can work only according to the rules laid down by one’s employer. In many 
other ways as well, modern man is strapped down by a network of rules and regulations (explicit 
or implicit) that frustrate many of his impulses and thus interfere with the power process. Most of 
these regulations cannot be dispensed with, because they are necessary for the functioning of 
industrial society. \vspace{\baselineskip}

72. Modern society is in certain respects extremely permissive. In matters that are irrelevant to 
the functioning of the system we can generally do what we please. We can believe in any religion 
(as long as it does not encourage behavior that is dangerous to the system). We can go to bed with 
anyone we like (as long as we practice “safe sex”). We can do anything we like as long as it is 
unimportant. But in all important matters the system tends increasingly to regulate our behavior. \vspace{\baselineskip}

73. Behavior is regulated not only through explicit rules and not only by the government. Control 
is often exercised through indirect coercion or through psychological pressure or manipulation, 
and by organizations other than the government, or by the system as a whole. Most large 
organizations use some form of propaganda14 to manipulate public attitudes or 
behavior. Propaganda\textsuperscript{14} is not limited to “commercials” and advertisements, and sometimes it is not 
even consciously intended as propaganda by the people who make it. For instance, the content of 
entertainment programming is a powerful form of propaganda. An example of indirect coercion: 
There is no law that says we have to go to work every day and follow our employer’s 
orders. Legally there is nothing to prevent us from going to live in the wild like primitive people 
or from going into business for ourselves. But in practice there is very little wild country left, and 
there is room in the economy for only a limited number of small business owners. Hence most of 
us can survive only as someone else’s employee. \vspace{\baselineskip}

74. We suggest that modern man’s obsession with longevity, and with maintaining physical vigor 
and sexual attractiveness to an advanced age, is a symptom of unfulfillment resulting from 
deprivation with respect to the power process. The “mid-life crisis” also is such a symptom. So 
is the lack of interest in having children that is fairly common in modern society but almost 
unheard-of in primitive societies. \vspace{\baselineskip}

75. In primitive societies life is a succession of stages. The needs and purposes of one stage 
having been fulfilled, there is no particular reluctance about passing on to the next stage. A young 
man goes through the power process by becoming a hunter, hunting not for sport or for fulfillment 
but to get meat that is necessary for food. (In young women the process is more complex, with 
greater emphasis on social power; we won’t discuss that here.) This phase having been successfully 
passed through, the young man has no reluctance about settling down to the responsibilities of 
raising a family. (In contrast, some modern people indefinitely postpone having children because 
they are too busy seeking some kind of “fulfillment.” We suggest that the fulfillment they need is 
adequate experience of the power process—with real goals instead of the artificial goals of 
surrogate activities.) Again, having successfully raised his children, going through the power 
process by providing them with the physical necessities, the primitive man feels that his work is 
done and he is prepared to accept old age (if he survives that long) and death. Many modern 
people, on the other hand, are disturbed by the prospect of physical deterioration and death, as is 
shown by the amount of effort they expend trying to maintain their physical condition, appearance 
and health. We argue that this is due to unfulfillment resulting from the fact that they have never 
put their physical powers to any practical use, have never gone through the power process using 
their bodies in a serious way. It is not the primitive man, who has used his body daily for practical 
purposes, who fears the deterioration of age, but the modern man, who has never had a practical 
use for his body beyond walking from his car to his house. It is the man whose need for the power 
process has been satisfied during his life who is best prepared to accept the end of that life. \vspace{\baselineskip}

76. In response to the arguments of this section someone will say, “Society must find a way to 
give people the opportunity to go through the power process.” This won’t work for those who need 
autonomy in the power process. For such people the value of the opportunity is destroyed by the 
very fact that society gives it to them. What they need is to find or make their own 
opportunities. As long as the system gives them their opportunities it still has them on a leash. To 
attain autonomy they must get off that leash. \newpage


\section{HOW SOME PEOPLE ADJUST}

\hspace{0.5cm} 77. Not everyone in industrial-technological society suffers from psychological problems. Some 
people even profess to be quite satisfied with society as it is. We now discuss some of the reasons 
why people differ so greatly in their response to modern society. \vspace{\baselineskip}

78. First, there doubtless are differences in the strength of the drive for power. Individuals with a 
weak drive for power may have relatively little need to go through the power process, or at least 
relatively little need for autonomy in the power process. These are docile types who would have 
been happy as plantation darkies in the Old South. (We don’t mean to sneer at the “plantation 
darkies” of the Old South. To their credit, most of the slaves were not content with their 
servitude. We do sneer at people who are content with servitude.) \vspace{\baselineskip}

79. Some people may have some exceptional drive, in pursuing which they satisfy their need for 
the power process. For example, those who have an unusually strong drive for social status may 
spend their whole lives climbing the status ladder without ever getting bored with that game. \vspace{\baselineskip} 

80. People vary in their susceptibility to advertising and marketing techniques. Some are so 
susceptible that, even if they make a great deal of money, they cannot satisfy their constant craving 
for the shiny new toys that the marketing industry dangles before their eyes. So they always feel 
hard-pressed financially even if their income is large, and their cravings are frustrated. \vspace{\baselineskip}

81. Some people have low susceptibility to advertising and marketing techniques. These are the 
people who aren’t interested in money. Material acquisition does not serve their need for the 
power process. \vspace{\baselineskip}

82. People who have medium susceptibility to advertising and marketing techniques are able to 
earn enough money to satisfy their craving for goods and services, but only at the cost of serious 
effort (putting in overtime, taking a second job, earning promotions, etc.). Thus material 
acquisition serves their need for the power process. But it does not necessarily follow that their 
need is fully satisfied. They may have insufficient autonomy in the power process (their work may 
consist of following orders) and some of their drives may be frustrated (e.g., security, 
aggression). (We are guilty of oversimplification in paragraphs 80-82 because we have assumed 
that the desire for material acquisition is entirely a creation of the advertising and marketing 
industry. Of course it’s not that simple.\textsuperscript{11} \vspace{\baselineskip}

83. Some people partly satisfy their need for power by identifying themselves with a powerful 
organization or mass movement. An individual lacking goals or power joins a movement or an 
organization, adopts its goals as his own, then works toward those goals. When some of the goals 
are attained, the individual, even though his personal efforts have played only an insignificant part 
in the attainment of the goals, feels (through his identification with the movement or organization) 
as if he had gone through the power process. This phenomenon was exploited by the fascists, nazis 
and communists. Our society uses it too, though less crudely. Example: Manuel Noriega was an 
irritant to the U.S. (goal: punish Noriega). The U.S. invaded Panama (effort) and punished Noriega 
(attainment of goal). Thus the U.S. went through the power process and many Americans, because 
of their identification with the U.S., experienced the power process vicariously. Hence the 
widespread public approval of the Panama invasion; it gave people a sense of power. We see the 
same phenomenon in armies, corporations, political parties, humanitarian organizations, religious 
or ideological movements. In particular, leftist movements tend to attract people who are seeking 
to satisfy their need for power. But for most people identification with a large organization or a 
mass movement does not fully satisfy the need for power.\textsuperscript{15} \vspace{\baselineskip} 

84. Another way in which people satisfy their need for the power process is through surrogate 
activities. As we explained in paragraphs 38-40, a surrogate activity is an activity that is directed 
toward an artificial goal that the individual pursues for the sake of the “fulfillment” that he gets 
from pursuing the goal, not because he needs to attain the goal itself. For instance, there is no 
practical motive for building enormous muscles, hitting a little ball into a hole or acquiring a 
complete series of postage stamps. Yet many people in our society devote themselves with passion 
to bodybuilding, golf or stamp-collecting. Some people are more “other-directed” than others, and 
therefore will more readily attach importance to a surrogate activity simply because the people 
around them treat it as important or because society tells them it is important. That is why some 
people get very serious about essentially trivial activities such as sports, or bridge, or chess, or 
arcane scholarly pursuits, whereas others who are more clear-sighted never see these things as 
anything but the surrogate activities that they are, and consequently never attach enough 
importance to them to satisfy their need for the power process in that way. It only remains to point 
out that in many cases a person’s way of earning a living is also a surrogate activity. Not a pure 
surrogate activity, since part of the motive for the activity is to gain the physical necessities and 
(for some people) social status and the luxuries that advertising makes them want. But many 
people put into their work far more effort than is necessary to earn whatever money and status they 
require, and this extra effort constitutes a surrogate activity. This extra effort, together with the 
emotional investment that accompanies it, is one of the most potent forces acting toward the 
continual development and perfecting of the system, with negative consequences for individual 
freedom (see paragraph 131). Especially, for the most creative scientists and engineers, work tends 
to be largely a surrogate activity. This point is so important that it deserves a separate discussion, 
which we shall give in a moment (paragraphs 87-92). \vspace{\baselineskip}

85. In this section we have explained how many people in modern society do satisfy their need 
for the power process to a greater or lesser extent. But we think that for the majority of people the 
need for the power process is not fully satisfied. In the first place, those who have an insatiable 
drive for status, or who get firmly “hooked” on a surrogate activity, or who identify strongly 
enough with a movement or organization to satisfy their need for power in that way, are 
exceptional personalities. Others are not fully satisfied with surrogate activities or by 
identification with an organization (see paragraphs 41, 64). In the second place, too much control 
is imposed by the system through explicit regulation or through socialization, which results in a 
deficiency of autonomy, and in frustration due to the impossibility of attaining certain goals and 
the necessity of restraining too many impulses. \vspace{\baselineskip} \newpage

86. But even if most people in industrial-technological society were well satisfied, we (FC) would 
still be opposed to that form of society, because (among other reasons) we consider it demeaning 
to fulfill one’s need for the power process through surrogate activities or through identification 
with an organization, rather than through pursuit of real goals. 


\section{THE MOTIVES OF SCIENTISTS}

\hspace{0.5cm} 87. Science and technology provide the most important examples of surrogate activities. Some 
scientists claim that they are motivated by “curiosity” or by a desire to “benefit humanity.” But it 
is easy to see that neither of these can be the principal motive of most scientists. As for “curiosity,” 
that notion is simply absurd. Most scientists work on highly specialized problems that are not the 
object of any normal curiosity. For example, is an astronomer, a mathematician or an entomologist 
curious about the properties of isopropyltrimethylmethane? Of course not. Only a chemist is 
curious about such a thing, and he is curious about it only because chemistry is his surrogate 
activity. Is the chemist curious about the appropriate classification of a new species of beetle? 
No. That question is of interest only to the entomologist, and he is interested in it only because 
entomology is his surrogate activity. If the chemist and the entomologist had to exert themselves 
seriously to obtain the physical necessities, and if that effort exercised their abilities in an 
interesting way but in some nonscientific pursuit, then they wouldn’t give a damn about 
isopropyltrimethylmethane or the classification of beetles. Suppose that lack of funds for 
postgraduate education had led the chemist to become an insurance broker instead of a chemist. In 
that case he would have been very interested in insurance matters but would have cared nothing 
about isopropyltrimethylmethane. In any case it is not normal to put into the satisfaction of mere 
curiosity the amount of time and effort that scientists put into their work. The “curiosity” 
explanation for the scientists’ motive just doesn’t stand up. \vspace{\baselineskip}

88. The “benefit of humanity” explanation doesn’t work any better. Some scientific work has no 
conceivable relation to the welfare of the human race—most of archaeology or comparative 
linguistics for example. Some other areas of science present obviously dangerous 
possibilities. Yet scientists in these areas are just as enthusiastic about their work as those who 
develop vaccines or study air pollution. Consider the case of Dr. Edward Teller, who had an 
obvious emotional involvement in promoting nuclear power plants. Did this involvement stem 
from a desire to benefit humanity? If so, then why didn’t Dr. Teller get emotional about other 
“humanitarian” causes? If he was such a humanitarian then why did he help to develop the H-
bomb? As with many other scientific achievements, it is very much open to question whether 
nuclear power plants actually do benefit humanity. Does the cheap electricity outweigh the 
accumulating waste and the risk of accidents? Dr. Teller saw only one side of the question. Clearly 
his emotional involvement with nuclear power arose not from a desire to “benefit humanity” but 
from a personal fulfillment he got from his work and from seeing it put to practical use. \vspace{\baselineskip} \newpage

89. The same is true of scientists generally. With possible rare exceptions, their motive is neither 
curiosity nor a desire to benefit humanity but the need to go through the power process: to have a 
goal (a scientific problem to solve), to make an effort (research) and to attain the goal (solution of 
the problem.) Science is a surrogate activity because scientists work mainly for the fulfillment they 
get out of the work itself. \vspace{\baselineskip}

90. Of course, it’s not that simple. Other motives do play a role for many scientists. Money and 
status for example. Some scientists may be persons of the type who have an insatiable drive for 
status (see paragraph 79) and this may provide much of the motivation for their work. No doubt 
the majority of scientists, like the majority of the general population, are more or less susceptible 
to advertising and marketing techniques and need money to satisfy their craving for goods and 
services. Thus science is not a pure surrogate activity. But it is in large part a surrogate activity. \vspace{\baselineskip}

91. Also, science and technology constitute a powerful mass movement, and many scientists 
gratify their need for power through identification with this mass movement (see paragraph 83). \vspace{\baselineskip}

92. Thus science marches on blindly, without regard to the real welfare of the human race or to 
any other standard, obedient only to the psychological needs of the scientists and of the government 
officials and corporation executives who provide the funds for research. 


\section{THE NATURE OF FREEDOM}

\hspace{0.5cm} 93. We are going to argue that industrial-technological society cannot be reformed in such a way 
as to prevent it from progressively narrowing the sphere of human freedom. But, because 
“freedom” is a word that can be interpreted in many ways, we must first make clear what kind of 
freedom we are concerned with. \vspace{\baselineskip}

94. By “freedom” we mean the opportunity to go through the power process, with real goals not 
the artificial goals of surrogate activities, and without interference, manipulation or supervision 
from anyone, especially from any large organization. Freedom means being in control (either as 
an individual or as a member of a small group) of the life-and-death issues of one’s existence: 
food, clothing, shelter and defense against whatever threats there may be in one’s 
environment. Freedom means having power; not the power to control other people but the power 
to control the circumstances of one’s own life. One does not have freedom if anyone else 
(especially a large organization) has power over one, no matter how benevolently, tolerantly and 
permissively that power may be exercised. It is important not to confuse freedom with mere 
permissiveness (see paragraph 72). \vspace{\baselineskip}

95. It is said that we live in a free society because we have a certain number of constitutionally 
guaranteed rights. But these are not as important as they seem. The degree of personal freedom 
that exists in a society is determined more by the economic and technological structure of the 
society than by its laws or its form of government.\textsuperscript{16} Most of the Indian nations of New England 
were monarchies, and many of the cities of the Italian Renaissance were controlled by dictators. But in reading about these societies one gets the impression that they allowed far more 
personal freedom than our society does. In part this was because they lacked efficient mechanisms 
for enforcing the ruler’s will: There were no modern, well-organized police forces, no rapid long-
distance communications, no surveillance cameras, no dossiers of information about the lives of 
average citizens. Hence it was relatively easy to evade control. \vspace{\baselineskip}

96. As for our constitutional rights, consider for example that of freedom of the press. We 
certainly don’t mean to knock that right; it is very important tool for limiting concentration of 
political power and for keeping those who do have political power in line by publicly exposing 
any misbehavior on their part. But freedom of the press is of very little use to the average citizen 
as an individual. The mass media are mostly under the control of large organizations that are 
integrated into the system. Anyone who has a little money can have something printed, or can 
distribute it on the Internet or in some such way, but what he has to say will be swamped by the 
vast volume of material put out by the media, hence it will have no practical effect. To make an 
impression on society with words is therefore almost impossible for most individuals and small 
groups. Take us (FC) for example. If we had never done anything violent and had submitted the 
present writings to a publisher, they probably would not have been accepted. If they had been 
been accepted and published, they probably would not have attracted many readers, because it’s 
more fun to watch the entertainment put out by the media than to read a sober essay. Even if these 
writings had had many readers, most of these readers would soon have forgotten what they had 
read as their minds were flooded by the mass of material to which the media expose them. In order 
to get our message before the public with some chance of making a lasting impression, we’ve had 
to kill people. \vspace{\baselineskip}

97. Constitutional rights are useful up to a point, but they do not serve to guarantee much more 
than what might be called the bourgeois conception of freedom. According to the bourgeois 
conception, a “free” man is essentially an element of a social machine and has only a certain set 
of prescribed and delimited freedoms; freedoms that are designed to serve the needs of the social 
machine more than those of the individual. Thus the bourgeois’s “free” man has economic 
freedom because that promotes growth and progress; he has freedom of the press because public 
criticism restrains misbehavior by political leaders; he has a right to a fair trial because 
imprisonment at the whim of the powerful would be bad for the system. This was clearly the 
attitude of Simon Bolivar. To him, people deserved liberty only if they used it to promote progress 
(progress as conceived by the bourgeois). Other bourgeois thinkers have taken a similar view of 
freedom as a mere means to collective ends. Chester C. Tan, “Chinese Political Thought in the 
Twentieth Century,” page 202, explains the philosophy of the Kuomintang leader Hu Han-min: 
“An individual is granted rights because he is a member of society and his community life requires 
such rights. By community Hu meant the whole society of the nation.” And on page 259 Tan 
states that according to Carsum Chang (Chang Chun-mai, head of the State Socialist Party in 
China) freedom had to be used in the interest of the state and of the people as a whole. But what 
kind of freedom does one have if one can use it only as someone else prescribes? FC’s conception 
of freedom is not that of Bolivar, Hu, Chang or other bourgeois theorists. The trouble with such 
theorists is that they have made the development and application of social theories their surrogate 
activity. Consequently the theories are designed to serve the needs of the theorists more than the 
needs of any people who may be unlucky enough to live in a society on which the theories are 
imposed. \vspace{\baselineskip}
 
98. One more point to be made in this section: It should not be assumed that a person has enough 
freedom just because he says he has enough. Freedom is restricted in part by psychological 
controls of which people are unconscious, and moreover many people’s ideas of what constitutes 
freedom are governed more by social convention than by their real needs. For example, it’s likely 
that many leftists of the oversocialized type would say that most people, including themselves, are 
socialized too little rather than too much, yet the oversocialized leftist pays a heavy psychological 
price for his high level of socialization. 


\section{SOME PRINCIPLES OF HISTORY}

\hspace{0.5cm} 99. Think of history as being the sum of two components: an erratic component that consists of 
unpredictable events that follow no discernible pattern, and a regular component that consists of 
long-term historical trends. Here we are concerned with the long-term trends. \vspace{\baselineskip}

100. \textit{First principle}. If a \textit{small} change is made that affects a long-term historical trend, then the 
effect of that change will almost always be transitory—the trend will soon revert to its original 
state. (Example: A reform movement designed to clean up political corruption in a society rarely 
has more than a short-term effect; sooner or later the reformers relax and corruption creeps back 
in. The level of political corruption in a given society tends to remain constant, or to change only 
slowly with the evolution of the society. Normally, a political cleanup will be permanent only if 
accompanied by widespread social changes; a \textit{small} change in the society won’t be enough.) If a
small change in a long-term historical trend appears to be permanent, it is only because the change 
acts in the direction in which the trend is already moving, so that the trend is not altered by only 
pushed a step ahead. \vspace{\baselineskip}

101. The first principle is almost a tautology. If a trend were not stable with respect to small 
changes, it would wander at random rather than following a definite direction; in other words it 
would not be a long-term trend at all. \vspace{\baselineskip}

102. \textit{Second principle}. If a change is made that is sufficiently large to alter permanently a long-
term historical trend, then it will alter the society as a whole. In other words, a society is a system 
in which all parts are interrelated, and you can’t permanently change any important part without 
changing all other parts as well. \vspace{\baselineskip}

103. \textit{Third principle}. If a change is made that is large enough to alter permanently a long-term 
trend, then the consequences for the society as a whole cannot be predicted in advance. (Unless 
various other societies have passed through the same change and have all experienced the same 
consequences, in which case one can predict on empirical grounds that another society that passes 
through the same change will be like to experience similar consequences.) \vspace{\baselineskip}

104. \textit{Fourth principle}. A new kind of society cannot be designed on paper. That is, you cannot 
plan out a new form of society in advance, then set it up and expect it to function as it was designed 
to do. \vspace{\baselineskip}

105. The third and fourth principles result from the complexity of human societies. A change in 
human behavior will affect the economy of a society and its physical environment; the economy 
 
will affect the environment and vice versa, and the changes in the economy and the environment 
will affect human behavior in complex, unpredictable ways; and so forth. The network of causes 
and effects is far too complex to be untangled and understood. \vspace{\baselineskip}

106. \textit{Fifth principle}. People do not consciously and rationally choose the form of their 
society. Societies develop through processes of social evolution that are not under rational human 
control. \vspace{\baselineskip} 

107. The fifth principle is a consequence of the other four. 108. To illustrate: By the first principle, generally speaking an attempt at social reform either acts 
in the direction in which the society is developing anyway (so that it merely accelerates a change 
that would have occurred in any case) or else it has only a transitory effect, so that the society soon 
slips back into its old groove. To make a lasting change in the direction of development of any 
important aspect of a society, reform is insufficient and revolution is required. (A revolution does 
not necessarily involve an armed uprising or the overthrow of a government.) By the second 
principle, a revolution never changes only one aspect of a society, it changes the whole society; 
and by the third principle changes occur that were never expected or desired by the 
revolutionaries. By the fourth principle, when revolutionaries or utopians set up a new kind of 
society, it never works out as planned. \vspace{\baselineskip}

109. The American Revolution does not provide a counterexample. The American “Revolution” 
was not a revolution in our sense of the word, but a war of independence followed by a rather far-
reaching political reform. The Founding Fathers did not change the direction of development of 
American society, nor did they aspire to do so. They only freed the development of American 
society from the retarding effect of British rule. Their political reform did not change any basic 
trend, but only pushed American political culture along its natural direction of 
development. British society, of which American society was an offshoot, had been moving for a 
long time in the direction of representative democracy. And prior to the War of Independence the 
Americans were already practicing a significant degree of representative democracy in the colonial 
assemblies. The political system established by the Constitution was modeled on the British 
system and on the colonial assemblies. With major alteration, to be sure—there is no doubt that 
the Founding Fathers took a very important step. But it was a step along the road that English-
speaking world was already traveling. The proof is that Britain and all of its colonies that were 
populated predominantly by people of British descent ended up with systems of representative 
democracy essentially similar to that of the United States. If the Founding Fathers had lost their 
nerve and declined to sign the Declaration of Independence, our way of life today would not have 
been significantly different. Maybe we would have had somewhat closer ties to Britain, and would 
have had a Parliament and Prime Minister instead of a Congress and President. No big deal. Thus 
the American Revolution provides not a counterexample to our principles but a good illustration 
of them. \vspace{\baselineskip} \newpage

110. Still, one has to use common sense in applying the principles. They are expressed in 
imprecise language that allows latitude for interpretation, and exceptions to them can be found. So 
we present these principles not as inviolable laws but as rules of thumb, or guides to thinking, that 
may provide a partial antidote to naive ideas about the future of society. The principles should be 
borne constantly in mind, and whenever one reaches a conclusion that conflicts with them one 
should carefully reexamine one’s thinking and retain the conclusion only if one has good, solid 
reasons for doing so. 


\section{INDUSTRIAL-TECHNOLOGICAL SOCIETY CANNOT BE REFORMED}

\hspace{0.5cm} 111. The foregoing principles help to show how hopelessly difficult it would be to reform the 
industrial system in such a way as to prevent it from progressively narrowing our sphere of 
freedom. There has been a consistent tendency, going back at least to the Industrial Revolution, 
for technology to strengthen the system at a high cost in individual freedom and local 
autonomy. Hence any change designed to protect freedom from technology would be contrary to 
a fundamental trend in the development of our society. Consequently, such a change either would 
be a transitory one—soon swamped by the tide of history or, if large enough to be permanent 
would alter the nature of our whole society. This by the first and second principles. Moreover, 
since society would be altered in a way that could not be predicted in advance (third principle) 
there would be great risk. Changes large enough to make a lasting difference in favor of freedom 
would not be initiated because it would be realized that they would gravely disrupt the system. So 
any attempts at reform would be too timid to be effective. Even if changes large enough to make 
a lasting difference were initiated, they would be retracted when their disruptive effects became 
apparent. Thus, permanent changes in favor of freedom could be brought about only by persons 
prepared to accept radical, dangerous and unpredictable alteration of the entire system. In other 
words by revolutionaries, not reformers. \vspace{\baselineskip}

112. People anxious to rescue freedom without sacrificing the supposed benefits of technology 
will suggest naive schemes for some new form of society that would reconcile freedom with 
technology. Apart from the fact that people who make such suggestions seldom propose any 
practical means by which the new form of society could be set up in the first place, it follows from 
the fourth principle that even if the new form of society could be once established, it either would 
collapse or would give results very different from those expected. \vspace{\baselineskip}

113. So even on very general grounds it seems highly improbable that any way of changing society 
could be found that would reconcile freedom with modern technology. In the next few sections 
we will give more specific reasons for concluding that freedom and technological progress are 
incompatible. 


\section{RESTRICTION OF FREEDOM IS UNAVOIDABLE IN INDUSTRIAL SOCIETY}

\hspace{0.5cm} 114. As explained in paragraphs 65-67, 70-73, modern man is strapped down by a network of 
rules and regulations, and his fate depends on the actions of persons remote from him whose 
decisions he cannot influence. This is not accidental or a result of the arbitrariness of arrogant 
bureaucrats. It is necessary and inevitable in any technologically advanced society. The system 
has to regulate human behavior closely in order to function. At work people have to do what they 
are told to do, otherwise production would be thrown into chaos. Bureaucracies have to be run 
according to rigid rules. To allow any substantial personal discretion to lower-level bureaucrats 
would disrupt the system and lead to charges of unfairness due to differences in the way individual 
bureaucrats exercised their discretion. It is true that some restrictions on our freedom could be 
eliminated, but generally speaking the regulation of our lives by large organizations is necessary 
for the functioning of industrial-technological society. The result is a sense of powerlessness on 
the part of the average person. It may be, however, that formal regulations will tend increasingly 
to be replaced by psychological tools that make us want to do what the system requires of 
us. (Propaganda, educational techniques, “mental health” programs, etc.) \vspace{\baselineskip}

115. The system has to force people to behave in ways that are increasingly remote from the 
natural pattern of human behavior. For example, the system needs scientists, mathematicians and 
engineers. It can’t function without them. So heavy pressure is put on children to excel in these 
fields. It isn’t natural for an adolescent human being to spend the bulk of his time sitting at a desk 
absorbed in study. A normal adolescent wants to spend his time in active contact with the real 
world. Among primitive peoples the things that children are trained to do tend to be in reasonable 
harmony with natural human impulses. Among the American Indians, for example, boys were 
trained in active outdoor pursuits—just the sort of thing that boys like. But in our society children 
are pushed into studying technical subjects, which most do grudgingly. \vspace{\baselineskip}

116. Because of the constant pressure that the system exerts to modify human behavior, there is a 
gradual increase in the number of people who cannot or will not adjust to society’s requirements: 
welfare leeches, youth gang members, cultists, anti-government rebels, radical environmentalist 
saboteurs, dropouts and resisters of various kinds. \vspace{\baselineskip}

117. In any technologically advanced society the individual’s fate must depend on decisions that 
he personally cannot influence to any great extent. A technological society cannot be broken down 
into small, autonomous communities, because production depends on the cooperation of very large 
numbers of people and machines. Such a society must be highly organized and decisions have to 
be made that affect very large numbers of people. When a decision affects, say, a million people, 
then each of the affected individuals has, on the average, only a one-millionth share in making the 
decision. What usually happens in practice is that decisions are made by public officials or 
corporation executives, or by technical specialists, but even when the public votes on a decision 
the number of voters ordinarily is too large for the vote of any one individual to be significant.\textsuperscript{17} 
Thus most individuals are unable to influence measurably the major decisions that affect their 
lives. There is no conceivable way to remedy this in a technologically advanced society. The 
system tries to “solve” this problem by using propaganda to make people want the decisions that 
have been made for them, but even if this “solution” were completely successful in making people 
feel better, it would be demeaning. \vspace{\baselineskip}

118. Conservatives and some others advocate more “local autonomy.” Local communities once 
did have autonomy, but such autonomy becomes less and less possible as local communities 
become more enmeshed with and dependent on large-scale systems like public utilities, computer 
networks, highway systems, the mass communications media, the modern health care 
system. Also operating against autonomy is the fact that technology applied in one location often 
affects people at other locations far way. Thus pesticide or chemical use near a creek may 
contaminate the water supply hundreds of miles downstream, and the greenhouse effect affects the 
whole world. \vspace{\baselineskip}
 
119. The system does not and cannot exist to satisfy human needs. Instead, it is human behavior 
that has to be modified to fit the needs of the system. This has nothing to do with the political or 
social ideology that may pretend to guide the technological system. It is not the fault of capitalism 
and it is not the fault of socialism. It is the fault of technology, because the system is guided not 
by ideology but by technical necessity\textsuperscript{18}. Of course the system does satisfy many human needs, 
but generally speaking it does this only to the extent that it is to the advantage of the system to do 
it. It is the needs of the system that are paramount, not those of the human being. For example, 
the system provides people with food because the system couldn’t function if everyone starved; it 
attends to people’s psychological needs whenever it can conveniently do so, because it couldn’t 
function if too many people became depressed or rebellious. But the system, for good, solid, 
practical reasons, must exert constant pressure on people to mold their behavior to the needs of the 
system. To much waste accumulating? The government, the media, the educational system, 
environmentalists, everyone inundates us with a mass of propaganda about recycling. Need more 
technical personnel? A chorus of voices exhorts kids to study science. No one stops to ask whether 
it is inhumane to force adolescents to spend the bulk of their time studying subjects most of them 
hate. When skilled workers are put out of a job by technical advances and have to undergo 
“retraining,” no one asks whether it is humiliating for them to be pushed around in this way. It is 
simply taken for granted that everyone must bow to technical necessity. and for good reason: If 
human needs were put before technical necessity there would be economic problems, 
unemployment, shortages or worse. The concept of “mental health” in our society is defined 
largely by the extent to which an individual behaves in accord with the needs of the system and 
does so without showing signs of stress. \vspace{\baselineskip}

120. Efforts to make room for a sense of purpose and for autonomy within the system are no better 
than a joke. For example, one company, instead of having each of its employees assemble only 
one section of a catalogue, had each assemble a whole catalogue, and this was supposed to give 
them a sense of purpose and achievement. Some companies have tried to give their employees 
more autonomy in their work, but for practical reasons this usually can be done only to a very 
limited extent, and in any case employees are never given autonomy as to ultimate goals—their 
“autonomous” efforts can never be directed toward goals that they select personally, but only 
toward their employer’s goals, such as the survival and growth of the company. Any company 
would soon go out of business if it permitted its employees to act otherwise. Similarly, in any 
enterprise within a socialist system, workers must direct their efforts toward the goals of the 
enterprise, otherwise the enterprise will not serve its purpose as part of the system. Once again, 
for purely technical reasons it is not possible for most individuals or small groups to have much 
autonomy in industrial society. Even the small-business owner commonly has only limited 
autonomy. Apart from the necessity of government regulation, he is restricted by the fact that he 
must fit into the economic system and conform to its requirements. For instance, when someone 
develops a new technology, the small-business person often has to use that technology whether he 
wants to or not, in order to remain competitive. 


\section{THE 'BAD' PARTS OF TECHNOLOGY CANNOT BE SEPARATED FROM THE 'GOOD' PARTS}

\hspace{0.5cm} 121. A further reason why industrial society cannot be reformed in favor of freedom is that modern 
technology is a unified system in which all parts are dependent on one another. You can’t get rid 
of the “bad” parts of technology and retain only the “good” parts. Take modern medicine, for 
example. Progress in medical science depends on progress in chemistry, physics, biology, 
computer science and other fields. Advanced medical treatments require expensive, high-tech 
equipment that can be made available only by a technologically progressive, economically rich 
society. Clearly you can’t have much Progress in medicine without the whole technologica
system and everything that goes with it. \vspace{\baselineskip}

122. Even if medical progress could be maintained without the rest of the technological system, 
it would by itself bring certain evils. Suppose for example that a cure for diabetes is 
discovered. People with a genetic tendency to diabetes will then be able to survive and reproduce 
as well as anyone else. Natural selection against genes for diabetes will cease and such genes will 
spread throughout the population. (This may be occurring to some extent already, since diabetes, 
while not curable, can be controlled through use of insulin.) The same thing will happen with many 
other diseases susceptibility to which is affected by genetic degradation of the population. The 
only solution will be some sort of eugenics program or extensive genetic engineering of human 
beings, so that man in the future will no longer be a creation of nature, or of chance, or of God 
(depending on your religious or philosophical opinions), but a manufactured product. \vspace{\baselineskip}

123. If you think that big government interferes in your life too much now, just wait till the 
government starts regulating the genetic constitution of your children. Such regulation will 
inevitably follow the introduction of genetic engineering of human beings, because the 
consequences of unregulated genetic engineering would be disastrous.\textsuperscript{19} \vspace{\baselineskip}

124. The usual response to such concerns is to talk about “medical ethics.” But a code of ethics 
would not serve to protect freedom in the face of medical progress; it would only make matters 
worse. A code of ethics applicable to genetic engineering would be in effect a means of regulating 
the genetic constitution of human beings. Somebody (probably the upper-middle class, mostly) 
would decide that such and such applications of genetic engineering were “ethical” and others 
were not, so that in effect they would be imposing their own values on the genetic constitution of 
the population at large. Even if a code of ethics were chosen on a completely democratic basis, 
the majority would be imposing their own values on any minorities who might have a different 
idea of what constituted an “ethical” use of genetic engineering. The only code of ethics that 
would truly protect freedom would be one that prohibited any genetic engineering of human 
beings, and you can be sure that no such code will ever be applied in a technological society. No 
code that reduced genetic engineering to a minor role could stand up for long, because the 
temptation presented by the immense power of biotechnology would be irresistible, especially 
since to the majority of people many of its applications will seem obviously and unequivocally 
good (eliminating physical and mental diseases, giving people the abilities they need to get along 
in today’s world). Inevitably, genetic engineering will be used extensively, but only in ways 
consistent with the needs of the industrial-technological system.\textsuperscript{20}


\section{TECHNOLOGY IS A MORE POWERFUL SOCIAL FORCE THAN THE ASPIRATION FOR FREEDOM}

\hspace{0.5cm} 125. It is not possible to make a lasting compromise between technology and freedom, because 
technology is by far the more powerful social force and continually encroaches on freedom through 
repeated compromises. Imagine the case of two neighbors, each of whom at the outset owns the 
same amount of land, but one of whom is more powerful than the other. The powerful one 
demands a piece of the other’s land. The weak one refuses. The powerful one says, “OK, let’s 
compromise. Give me half of what I asked.” The weak one has little choice but to give in. Some 
time later the powerful neighbor demands another piece of land, again there is a compromise, and 
so forth. By forcing a long series of compromises on the weaker man, the powerful one eventually 
gets all of his land. So it goes in the conflict between technology and freedom. \vspace{\baselineskip}

126. Let us explain why technology is a more powerful social force than the aspiration for 
freedom. \vspace{\baselineskip}

127. A technological advance that appears not to threaten freedom often turns out to threaten it 
very seriously later on. For example, consider motorized transport. A walking man formerly 
could go where he pleased, go at his own pace without observing any traffic regulations, and was 
independent of technological support-systems. When motor vehicles were introduced they 
appeared to increase man’s freedom. They took no freedom away from the walking man, no one 
had to have an automobile if he didn’t want one, and anyone who did choose to buy an automobile 
could travel much faster and farther than a walking man. But the introduction of motorized 
transport soon changed society in such a way as to restrict greatly man’s freedom of 
locomotion. When automobiles became numerous, it became necessary to regulate their use 
extensively. In a car, especially in densely populated areas, one cannot just go where one likes at 
one’s own pace; one’s movement is governed by the flow of traffic and by various traffic 
laws. One is tied down by various obligations: license requirements, driver test, renewing 
registration, insurance, maintenance required for safety, monthly payments on purchase 
price. Moreover, the use of motorized transport is no longer optional. Since the introduction of 
motorized transport the arrangement of our cities has changed in such a way that the majority of 
people no longer live within walking distance of their place of employment, shopping areas and 
recreational opportunities, so that they have to depend on the automobile for transportation. Or 
else they must use public transportation, in which case they have even less control over their own 
movement than when driving a car. Even the walker’s freedom is now greatly restricted. In the 
city he continually has to stop to wait for traffic lights that are designed mainly to serve auto 
traffic. In the country, motor traffic makes it dangerous and unpleasant to walk along the 
highway. (Note this important point that we have just illustrated with the case of motorized 
transport: When a new item of technology is introduced as an option that an individual can accept 
or not as he chooses, it does not necessarily remain optional. In many cases the new technology 
changes society in such a way that people eventually find themselves forced to use it.) \vspace{\baselineskip}

128. While technological progress \textit{as a whole} continually narrows our sphere of freedom, each 
new technical advance \textit{considered by itself} appears to be desirable. Electricity, indoor plumbing, 
rapid long-distance communications... how could one argue against any of these things, or against 
any other of the innumerable technical advances that have made modern society? It would have 
been absurd to resist the introduction of the telephone, for example. It offered many advantages 
and no disadvantages. Yet, as we explained in paragraphs 59-76, all these technical advances 
taken together have created a world in which the average man’s fate is no longer in his own hands 
or in the hands of his neighbors and friends, but in those of politicians, corporation executives and 
remote, anonymous technicians and bureaucrats whom he as an individual has no power to 
influence.\textsuperscript{21} The same process will continue in the future. Take genetic engineering, for 
example. Few people will resist the introduction of a genetic technique that eliminates a hereditary 
disease. It does no apparent harm and prevents much suffering. Yet a large number of genetic 
improvements taken together will make the human being into an engineered product rather than a 
free creation of chance (or of God, or whatever, depending on your religious beliefs). \vspace{\baselineskip}

129. Another reason why technology is such a powerful social force is that, within the context of 
a given society, technological progress marches in only one direction; it can never be 
reversed. Once a technical innovation has been introduced, people usually become dependent on 
it, so that they can never again do without it, unless it is replaced by some still more advanced 
innovation. Not only do people become dependent as individuals on a new item of technology, 
but, even more, the system as a whole becomes dependent on it. (Imagine what would happen to 
the system today if computers, for example, were eliminated.) Thus the system can move in only 
one direction, toward greater technologization. Technology repeatedly forces freedom to take a 
step back, but technology can never take a step back—short of the overthrow of the whole 
technological system. \vspace{\baselineskip}

130. Technology advances with great rapidity and threatens freedom at many different points at 
the same time (crowding, rules and regulations, increasing dependence of individuals on large 
organizations, propaganda and other psychological techniques, genetic engineering, invasion of 
privacy through surveillance devices and computers, etc.). To hold back any one of the threats to 
freedom would require a long and difficult social struggle. Those who want to protect freedom 
are overwhelmed by the sheer number of new attacks and the rapidity with which they develop, 
hence they become apathetic and no longer resist. To fight each of the threats separately would 
be futile. Success can be hoped for only by fighting the technological system as a whole; but that 
is revolution, not reform. \vspace{\baselineskip}

131. Technicians (we use this term in its broad sense to describe all those who perform a 
specialized task that requires training) tend to be so involved in their work (their surrogate activity) 
that when a conflict arises between their technical work and freedom, they almost always decide 
in favor of their technical work. This is obvious in the case of scientists, but it also appears 
elsewhere: educators, humanitarian groups, conservation organizations do not hesitate to use 
propaganda or other psychological techniques to help them achieve their laudable 
ends. Corporations and government agencies, when they find it useful, do not hesitate to collect 
information about individuals without regard to their privacy. Law enforcement agencies are 
frequently inconvenienced by the constitutional rights of suspects and often of completely innocent 
persons, and they do whatever they can do legally (or sometimes illegally) to restrict or circumvent 
those rights. Most of these educators, government officials and law officers believe in freedom, 
privacy and constitutional rights, but when these conflict with their work, they usually feel that 
their work is more important. \vspace{\baselineskip}

132. It is well known that people generally work better and more persistently when striving for a 
reward than when attempting to avoid a punishment or negative outcome. Scientists and other 
technicians are motivated mainly by the rewards they get through their work. But those who 
oppose technological invasions of freedom are working to avoid a negative outcome, consequently 
there are few who work persistently and well at this discouraging task. If reformers ever achieved 
 
a single victory that seemed to set up a solid barrier against further erosion of freedom through 
technical progress, most would tend to relax and turn their attention to more agreeable 
pursuits. But the scientists would remain busy in their laboratories, and technology as it progresses 
would find ways, in spite of any barriers, to exert more and more control over individuals and 
make them always more dependent on the system. \vspace{\baselineskip}

133. No social arrangements, whether laws, institutions, customs or ethical codes, can provide 
permanent protection against technology. History shows that all social arrangements are 
transitory; they all change or break down eventually. But technological advances are permanent 
within the context of a given civilization. Suppose for example that it were possible to arrive at 
some social arrangements that would prevent genetic engineering from being applied to human 
beings, or prevent it from being applied in such a way as to threaten freedom and dignity. Still, 
the technology would remain waiting. Sooner or later the social arrangement would break 
down. Probably sooner, given the pace of change in our society. Then genetic engineering would 
begin to invade our sphere of freedom, and this invasion would be irreversible (short of a 
breakdown of technological civilization itself). Any illusions about achieving anything permanent 
through social arrangements should be dispelled by what is currently happening with 
environmental legislation. A few years ago its seemed that there were secure legal barriers 
preventing at least some of the worst forms of environmental degradation. A change in the political 
wind, and those barriers begin to crumble. \vspace{\baselineskip} \newline

134. For all of the foregoing reasons, technology is a more powerful social force than the 
aspiration for freedom. But this statement requires an important qualification. It appears that 
during the next several decades the industrial-technological system will be undergoing severe 
stresses due to economic and environmental problems, and especially due to problems of human 
behavior (alienation, rebellion, hostility, a variety of social and psychological difficulties). We 
hope that the stresses through which the system is likely to pass will cause it to break down, or at 
least will weaken it sufficiently so that a revolution against it becomes possible. If such a 
revolution occurs and is successful, then at that particular moment the aspiration for freedom will 
have proved more powerful than technology. \vspace{\baselineskip}

135. In paragraph 125 we used an analogy of a weak neighbor who is left destitute by a strong 
neighbor who takes all his land by forcing on him a series of compromises. But suppose now that 
the strong neighbor gets sick, so that he is unable to defend himself. The weak neighbor can force 
the strong one to give him his land back, or he can kill him. If he lets the strong man survive and 
only forces him to give the land back, he is a fool, because when the strong man gets well he will 
again take all the land for himself. The only sensible alternative for the weaker man is to kill the 
strong one while he has the chance. In the same way, while the industrial system is sick we must 
destroy it. If we compromise with it and let it recover from its sickness, it will eventually wipe 
out all of our freedom. 


\section{SIMPLER SOCIAL PROBLEMS HAVE PROVED INTRACTABLE}

\hspace{0.5cm} 136. If anyone still imagines that it would be possible to reform the system in such a way as to 
protect freedom from technology, let him consider how clumsily and for the most part 
unsuccessfully our society has dealt with other social problems that are far more simple and
straightforward. Among other things, the system has failed to stop environmental degradation, 
political corruption, drug trafficking or domestic abuse. \vspace{\baselineskip}

137. Take our environmental problems, for example. Here the conflict of values is 
straightforward: economic expedience now versus saving some of our natural resources for our 
grandchildren.\textsuperscript{22} But on this subject we get only a lot of blather and obfuscation from the people 
who have power, and nothing like a clear, consistent line of action, and we keep on piling up 
environmental problems that our grandchildren will have to live with. Attempts to resolve the 
environmental issue consist of struggles and compromises between different factions, some of 
which are ascendant at one moment, others at another moment. The line of struggle changes with 
the shifting currents of public opinion. This is not a rational process, nor is it one that is likely to 
lead to a timely and successful solution to the problem. Major social problems, if they get “solved” 
at all, are rarely or never solved through any rational, comprehensive plan. They just work 
themselves out through a process in which various competing groups pursuing their own (usually 
short-term) self-interest\textsuperscript{23} arrive (mainly by luck) at some more or less stable modus vivendi. In 
fact, the principles we formulated in paragraphs 100-106 make it seem doubtful that rational long-
term social planning can ever be successful. \vspace{\baselineskip}

138. Thus it is clear that the human race has at best a very limited capacity for solving even 
relatively straightforward social problems. How then is it going to solve the far more difficult and 
subtle problem of reconciling freedom with technology? Technology presents clear-cut material 
advantages, whereas freedom is an abstraction that means different things to different people and 
its loss is easily obscured by propaganda and fancy talk. \vspace{\baselineskip}

139. And note this important difference: It is conceivable that our environmental problems (for 
example) may some day be settled through a rational, comprehensive plan, but if this happens it 
will be only because it is in the long-term interest of the system to solve these problems. But it is 
not in the interest of the system to preserve freedom or small-group autonomy. On the contrary, 
it is in the interest of the system to bring human behavior under control to the greatest possible 
extent.\textsuperscript{24} Thus, while practical considerations may eventually force the system to take a rational, 
prudent approach to environmental problems, equally practical considerations will force the 
system to regulate human behavior ever more closely (preferably by indirect means that will 
disguise the encroachment on freedom). This isn’t just our opinion. Eminent social scientists 
(e.g. James Q. Wilson) have stressed the importance of “socializing” people more effectively. 


\section{REVOLUTION IS EASIER THAN REFORM}

\hspace{0.5cm} 140. We hope we have convinced the reader that the system cannot be reformed in such a way as 
to reconcile freedom with technology. The only way out is to dispense with the industrial-
technological system altogether. This implies revolution, not necessarily an armed uprising, but 
certainly a radical and fundamental change in the nature of society. \vspace{\baselineskip} 

141. People tend to assume that because a revolution involves a much greater change than reform 
does, it is more difficult to bring about than reform is. Actually, under certain circumstances 
revolution is much easier than reform. The reason is that a revolutionary movement can inspire 
an intensity of commitment that a reform movement cannot inspire. A reform movement merely 
offers to solve a particular social problem. A revolutionary movement offers to solve all problems 
at one stroke and create a whole new world; it provides the kind of ideal for which people will take 
great risks and make great sacrifices. For this reasons it would be much easier to overthrow the 
whole technological system than to put effective, permanent restraints on the development or 
application of any one segment of technology, such as genetic engineering, for example. Not many 
people will devote themselves with single-minded passion to imposing and maintaining restraints 
on genetic engineering, but under suitable conditions large numbers of people may devote 
themselves passionately to a revolution against the industrial-technological system. As we noted 
in paragraph 132, reformers seeking to limit certain aspects of technology would be working to 
avoid a negative outcome. But revolutionaries work to gain a powerful reward—fulfillment of 
their revolutionary vision—and therefore work harder and more persistently than reformers do. \vspace{\baselineskip} \newpage

142. Reform is always restrained by the fear of painful consequences if changes go too far. But 
once a revolutionary fever has taken hold of a society, people are willing to undergo unlimited 
hardships for the sake of their revolution. This was clearly shown in the French and Russian 
Revolutions. It may be that in such cases only a minority of the population is really committed to 
the revolution, but this minority is sufficiently large and active so that it becomes the dominant 
force in society. We will have more to say about revolution in paragraphs 180-205. 


\section{CONTROL OF HUMAN BEHAVIOR}

\hspace{0.5cm} 143. Since the beginning of civilization, organized societies have had to put pressures on human 
beings for the sake of the functioning of the social organism. The kinds of pressures vary greatly 
from one society to another. Some of the pressures are physical (poor diet, excessive labor, 
environmental pollution), some are psychological (noise, crowding, forcing human behavior into 
the mold that society requires). In the past, human nature has been approximately constant, or at 
any rate has varied only within certain bounds. Consequently, societies have been able to push 
people only up to certain limits. When the limit of human endurance has been passed, things start 
going wrong: rebellion, or crime, or corruption, or evasion of work, or depression and other mental 
problems, or an elevated death rate, or a declining birth rate or something else, so that either the 
society breaks down, or its functioning becomes too inefficient and it is (quickly or gradually, 
through conquest, attrition or evolution) replaced by some more efficient form of society.\textsuperscript{25} \vspace{\baselineskip}

144. Thus human nature has in the past put certain limits on the development of societies. People 
could be pushed only so far and no farther. But today this may be changing, because modern 
technology is developing ways of modifying human beings. \vspace{\baselineskip}

145. Imagine a society that subjects people to conditions that make them terribly unhappy, then 
gives them drugs to take away their unhappiness. Science fiction? It is already happening to some 
extent in our own society. It is well known that the rate of clinical depression has been greatly 
increasing in recent decades. We believe that this is due to disruption of the power process, as 
explained in paragraphs 59-76. But even if we are wrong, the increasing rate of depression is 
certainly the result of some conditions that exist in today’s society. Instead of removing the 
conditions that make people depressed, modern society gives them antidepressant drugs. In effect, 
antidepressants are a means of modifying an individual’s internal state in such a way as to enable 
him to tolerate social conditions that he would otherwise find intolerable. (Yes, we know that 
depression is often of purely genetic origin. We are referring here to those cases in which 
environment plays the predominant role.) \vspace{\baselineskip}

146. Drugs that affect the mind are only one example of the new methods of controlling human 
behavior that modern society is developing. Let us look at some of the other methods. \vspace{\baselineskip} \newpage

147. To start with, there are the techniques of surveillance. Hidden video cameras are now used 
in most stores and in many other places, computers are used to collect and process vast amounts 
of information about individuals. Information so obtained greatly increases the effectiveness of 
physical coercion (i.e., law enforcement).\textsuperscript{26} Then there are the methods of propaganda, for which 
the mass communication media provide effective vehicles. Efficient techniques have been 
developed for winning elections, selling products, influencing public opinion. The entertainment 
industry serves as an important psychological tool of the system, possibly even when it is dishing 
out large amounts of sex and violence. Entertainment provides modern man with an essential 
means of escape. While absorbed in television, videos, etc., he can forget stress, anxiety, 
frustration, dissatisfaction. Many primitive peoples, when they don’t have work to do, are quite 
content to sit for hours at a time doing nothing at all, because they are at peace with themselves 
and their world. But most modern people must be constantly occupied or entertained, otherwise 
they get “bored,” i.e., they get fidgety, uneasy, irritable. \vspace{\baselineskip}

148. Other techniques strike deeper than the foregoing. Education is no longer a simple affair of 
paddling a kid’s behind when he doesn’t know his lessons and patting him on the head when he 
does know them. It is becoming a scientific technique for controlling the child’s 
development. Sylvan Learning Centers, for example, have had great success in motivating 
children to study, and psychological techniques are also used with more or less success in many 
conventional schools. “Parenting” techniques that are taught to parents are designed to make 
children accept fundamental values of the system and behave in ways that the system finds 
desirable. “Mental health” programs, “intervention” techniques, psychotherapy and so forth are 
ostensibly designed to benefit individuals, but in practice they usually serve as methods for 
inducing individuals to think and behave as the system requires. (There is no contradiction here; 
an individual whose attitudes or behavior bring him into conflict with the system is up against a 
force that is too powerful for him to conquer or escape from, hence he is likely to suffer from 
stress, frustration, defeat. His path will be much easier if he thinks and behaves as the system 
requires. In that sense the system is acting for the benefit of the individual when it brainwashes 
him into conformity.) Child abuse in its gross and obvious forms is disapproved in most if not all 
cultures. Tormenting a child for a trivial reason or no reason at all is something that appalls almost 
everyone. But many psychologists interpret the concept of abuse much more broadly. Is spanking, 
when used as part of a rational and consistent system of discipline, a form of abuse? The question 
will ultimately be decided by whether or not spanking tends to produce behavior that makes a 
person fit in well with the existing system of society. In practice, the word “abuse” tends to be 
interpreted to include any method of child-rearing that produces behavior inconvenient for the 
system. Thus, when they go beyond the prevention of obvious, senseless cruelty, programs for 
preventing “child abuse” are directed toward the control of human behavior on behalf of the 
system. \vspace{\baselineskip}
 
149. Presumably, research will continue to increase the effectiveness of psychological techniques 
for controlling human behavior. But we think it is unlikely that psychological techniques alone 
will be sufficient to adjust human beings to the kind of society that technology is 
creating. Biological methods probably will have to be used. We have already mentioned the use 
of drugs in this connection. Neurology may provide other avenues for modifying the human 
mind. Genetic engineering of human beings is already beginning to occur in the form of “gene 
therapy,” and there is no reason to assume that such methods will not eventually be used to modify 
those aspects of the body that affect mental functioning. \vspace{\baselineskip}

150. As we mentioned in paragraph 134, industrial society seems likely to be entering a period of 
severe stress, due in part to problems of human behavior and in part to economic and 
environmental problems. And a considerable proportion of the system’s economic and 
environmental problems result from the way human beings behave. Alienation, low self-esteem, 
depression, hostility, rebellion; children who won’t study, youth gangs, illegal drug use, rape, child 
abuse, other crimes, unsafe sex, teen pregnancy, population growth, political corruption, race 
hatred, ethnic rivalry, bitter ideological conflict (e.g., pro-choice vs. pro-life), political extremism, 
terrorism, sabotage, anti-government groups, hate groups. All these threaten the very survival of 
the system. The system will therefore be forced to use every practical means of controlling human 
behavior. \vspace{\baselineskip}

151. The social disruption that we see today is certainly not the result of mere chance. It can only 
be a result of the conditions of life that the system imposes on people. (We have argued that the 
most important of these conditions is disruption of the power process.) If the systems succeeds in 
imposing sufficient control over human behavior to assure its own survival, a new watershed in 
human history will have been passed. Whereas formerly the limits of human endurance have 
imposed limits on the development of societies (as we explained in Paragraphs 143, 144), 
industrial-technological society will be able to pass those limits by modifying human beings, 
whether by psychological methods or biological methods or both. In the future, social systems 
will not be adjusted to suit the needs of human beings. Instead, human being will be adjusted to 
suit the needs of the system.\textsuperscript{27} \vspace{\baselineskip}

152. Generally speaking, technological control over human behavior will probably not be 
introduced with a totalitarian intention or even through a conscious desire to restrict human 
freedom.\textsuperscript{28} Each new step in the assertion of control over the human mind will be taken as a rational 
response to a problem that faces society, such as curing alcoholism, reducing the crime rate or 
inducing young people to study science and engineering. In many cases there will be a 
humanitarian justification. For example, when a psychiatrist prescribes an anti-depressant for a 
depressed patient, he is clearly doing that individual a favor. It would be inhumane to withhold 
the drug from someone who needs it. When Parents send their children to Sylvan Learning Centers 
to have them manipulated into becoming enthusiastic about their studies, they do so from concern 
for their children’s welfare. It may be that some of these parents wish that one didn’t have to have 
specialized training to get a job and that their kid didn’t have to be brainwashed into becoming a 
computer nerd. But what can they do? They can’t change society, and their child may be 
unemployable if he doesn’t have certain skills. So they send him to Sylvan. \vspace{\baselineskip}
 
153. Thus control over human behavior will be introduced not by a calculated decision of the 
authorities but through a process of social evolution (rapid evolution, however). The process will 
be impossible to resist, because each advance, considered by itself, will appear to be beneficial, or 
at least the evil involved in making the advance will appear to be beneficial, or at least the evil 
involved in making the advance will seem to be less than that which would result from not making 
it (see paragraph 127). Propaganda for example is used for many good purposes, such as 
discouraging child abuse or race hatred.\textsuperscript{14} Sex education is obviously useful, yet the effect of 
sex education (to the extent that it is successful) is to take the shaping of sexual attitudes away 
from the family and put it into the hands of the state as represented by the public school system. \vspace{\baselineskip}

154. Suppose a biological trait is discovered that increases the likelihood that a child will grow up 
to be a criminal, and suppose some sort of gene therapy can remove this trait.\textsuperscript{29} Of course most 
parents whose children possess the trait will have them undergo the therapy. It would be inhumane 
to do otherwise, since the child would probably have a miserable life if he grew up to be a 
criminal. But many or most primitive societies have a low crime rate in comparison with that of 
our society, even though they have neither high-tech methods of child-rearing nor harsh systems 
of punishment. Since there is no reason to suppose that more modern men than primitive men 
have innate predatory tendencies, the high crime rate of our society must be due to the pressures 
that modern conditions put on people, to which many cannot or will not adjust. Thus a treatment 
designed to remove potential criminal tendencies is at least in part a way of re-engineering people 
so that they suit the requirements of the system. \vspace{\baselineskip}

155. Our society tends to regard as a “sickness” any mode of thought or behavior that is 
inconvenient for the system, and this is plausible because when an individual doesn’t fit into the 
system it causes pain to the individual as well as problems for the system. Thus the manipulation 
of an individual to adjust him to the system is seen as a “cure” for a “sickness” and therefore as 
good. \vspace{\baselineskip}

156. In paragraph 127 we pointed out that if the use of a new item of technology is initially 
optional, it does not necessarily remain optional, because the new technology tends to change 
society in such a way that it becomes difficult or impossible for an individual to function without 
using that technology. This applies also to the technology of human behavior. In a world in which 
most children are put through a program to make them enthusiastic about studying, a parent will 
almost be forced to put his kid through such a program, because if he does not, then the kid will 
grow up to be, comparatively speaking, an ignoramus and therefore unemployable. Or suppose a 
biological treatment is discovered that, without undesirable side-effects, will greatly reduce the 
psychological stress from which so many people suffer in our society. If large numbers of people 
choose to undergo the treatment, then the general level of stress in society will be reduced, so that 
it will be possible for the system to increase the stress-producing pressures. This will lead more 
people to undergo the treatment; and so forth, so that eventually the pressures may become so 
heavy that few people will be able to survive without undergoing the stress-reducing treatment. In 
fact, something like this seems to have happened already with one of our society’s most important 
psychological tools for enabling people to reduce (or at least temporarily escape from) stress, 
namely, mass entertainment (see paragraph 147). Our use of mass entertainment is “optional”: No 
law requires us to watch television, listen to the radio, read magazines. Yet mass entertainment is 
a means of escape and stress-reduction on which most of us have become dependent. Everyone 
complains about the trashiness of television, but almost everyone watches it. A few have kicked 
the TV habit, but it would be a rare person who could get along today without using any form of 
mass entertainment. (Yet until quite recently in human history most people got along very nicely 
with no other entertainment than that which each local community created for itself.) Without the 
entertainment industry the system probably would not have been able to get away with putting as 
much stress-producing pressure on us as it does. \vspace{\baselineskip}

157. Assuming that industrial society survives, it is likely that technology will eventually acquire 
something approaching complete control over human behavior. It has been established beyond 
any rational doubt that human thought and behavior have a largely biological basis. As 
experimenters have demonstrated, feelings such as hunger, pleasure, anger and fear can be turned 
on and off by electrical stimulation of appropriate parts of the brain. Memories can be destroyed 
by damaging parts of the brain or they can be brought to the surface by electrical 
stimulation. Hallucinations can be induced or moods changed by drugs. There may or may not 
be an immaterial human soul, but if there is one it clearly is less powerful that the biological 
mechanisms of human behavior. For if that were not the case then researchers would not be able 
so easily to manipulate human feelings and behavior with drugs and electrical currents. \vspace{\baselineskip}

158. It presumably would be impractical for all people to have electrodes inserted in their heads 
so that they could be controlled by the authorities. But the fact that human thoughts and feelings 
are so open to biological intervention shows that the problem of controlling human behavior is 
mainly a technical problem; a problem of neurons, hormones and complex molecules; the kind of 
problem that is accessible to scientific attack. Given the outstanding record of our society in 
solving technical problems, it is overwhelmingly probable that great advances will be made in the 
control of human behavior. \vspace{\baselineskip}

159. Will public resistance prevent the introduction of technological control of human behavior? 
It certainly would if an attempt were made to introduce such control all at once. But since 
technological control will be introduced through a long sequence of small advances, there will be 
no rational and effective public resistance. (See paragraphs 127, 132, 153.) \vspace{\baselineskip}

160. To those who think that all this sounds like science fiction, we point out that yesterday’s 
science fiction is today’s fact. The Industrial Revolution has radically altered man’s environment 
and way of life, and it is only to be expected that as technology is increasingly applied to the human 
body and mind, man himself will be altered as radically as his environment and way of life have 
been. 


\section{HUMAN RACE AT A CROSSROADS}

\hspace{0.5cm} 161. But we have gotten ahead of our story. It is one thing to develop in the laboratory a series 
of psychological or biological techniques for manipulating human behavior and quite another to 
integrate these techniques into a functioning social system. The latter problem is the more difficult 
of the two. For example, while the techniques of educational psychology doubtless work quite 
well in the “lab schools” where they are developed, it is not necessarily easy to apply them 
effectively throughout our educational system. We all know what many of our schools are 
like. The teachers are too busy taking knives and guns away from the kids to subject them to the 
latest techniques for making them into computer nerds. Thus, in spite of all its technical advances 
relating to human behavior, the system to date has not been impressively successful in controlling 
human beings. The people whose behavior is fairly well under the control of the system are those 
of the type that might be called “bourgeois.” But there are growing numbers of people who in one 
way or another are rebels against the system: welfare leaches, youth gangs, cultists, satanists, nazis, 
radical environmentalists, militiamen, etc. \vspace{\baselineskip}

162. The system is currently engaged in a desperate struggle to overcome certain problems that 
threaten its survival, among which the problems of human behavior are the most important. If the 
system succeeds in acquiring sufficient control over human behavior quickly enough, it will 
probably survive. Otherwise it will break down. We think the issue will most likely be resolved 
within the next several decades, say 40 to 100 years. \vspace{\baselineskip}

163. Suppose the system survives the crisis of the next several decades. By that time it will have 
to have solved, or at least brought under control, the principal problems that confront it, in 
particular that of “socializing” human beings; that is, making people sufficiently docile so that 
their behavior no longer threatens the system. That being accomplished, it does not appear that 
there would be any further obstacle to the development of technology, and it would presumably 
advance toward its logical conclusion, which is complete control over everything on Earth, 
including human beings and all other important organisms. The system may become a unitary, 
monolithic organization, or it may be more or less fragmented and consist of a number of 
organizations coexisting in a relationship that includes elements of both cooperation and 
competition, just as today the government, the corporations and other large organizations both 
cooperate and compete with one another. Human freedom mostly will have vanished, because 
individuals and small groups will be impotent vis-a-vis large organizations armed with 
supertechnology and an arsenal of advanced psychological and biological tools for manipulating 
human beings, besides instruments of surveillance and physical coercion. Only a small number of 
people will have any real power, and even these probably will have only very limited freedom, 
because their behavior too will be regulated; just as today our politicians and corporation 
executives can retain their positions of power only as long as their behavior remains within certain 
fairly narrow limits. \vspace{\baselineskip}

164. Don’t imagine that the systems will stop developing further techniques for controlling human 
beings and nature once the crisis of the next few decades is over and increasing control is no longer 
necessary for the system’s survival. On the contrary, once the hard times are over the system will 
increase its control over people and nature more rapidly, because it will no longer be hampered by 
difficulties of the kind that it is currently experiencing. Survival is not the principal motive for 
extending control. As we explained in paragraphs 87-90, technicians and scientists carry on their 
work largely as a surrogate activity; that is, they satisfy their need for power by solving technical 
problems. They will continue to do this with unabated enthusiasm, and among the most interesting 
and challenging problems for them to solve will be those of understanding the human body and 
mind and intervening in their development. For the “good of humanity,” of course. \vspace{\baselineskip}

165. But suppose on the other hand that the stresses of the coming decades prove to be too much 
for the system. If the system breaks down there may be a period of chaos, a “time of troubles” 
such as those that history has recorded at various epochs in the past. It is impossible to predict 
what would emerge from such a time of troubles, but at any rate the human race would be given a 
new chance. The greatest danger is that industrial society may begin to reconstitute itself within 
the first few years after the breakdown. Certainly there will be many people (power-hungry types 
especially) who will be anxious to get the factories running again. \vspace{\baselineskip}

166. Therefore two tasks confront those who hate the servitude to which the industrial system is 
reducing the human race. First, we must work to heighten the social stresses within the system so 
as to increase the likelihood that it will break down or be weakened sufficiently so that a revolution 
against it becomes possible. Second, it is necessary to develop and propagate an ideology that 
opposes technology and the industrial system. Such an ideology can become the basis for a 
revolution against industrial society if and when the system becomes sufficiently weakened. And 
such an ideology will help to assure that, if and when industrial society breaks down, its remnants 
will be smashed beyond repair, so that the system cannot be reconstituted. The factories should 
be destroyed, technical books burned, etc. 


\section{HUMAN SUFFERING}

\hspace{0.5cm} 167. The industrial system will not break down purely as a result of revolutionary action. It will 
not be vulnerable to revolutionary attack unless its own internal problems of development lead it 
into very serious difficulties. So if the system breaks down it will do so either spontaneously, or 
through a process that is in part spontaneous but helped along by revolutionaries. If the breakdown 
is sudden, many people will die, since the world’s population has become so overblown that it 
cannot even feed itself any longer without advanced technology. Even if the breakdown is gradual 
enough so that reduction of the population can occur more through lowering of the birth rate than 
through elevation of the death rate, the process of de-industrialization probably will be very chaotic 
and involve much suffering. It is naive to think it likely that technology can be phased out in a 
smoothly managed, orderly way, especially since the technophiles will fight stubbornly at every 
step. Is it therefore cruel to work for the breakdown of the system? Maybe, but maybe not. In the 
first place, revolutionaries will not be able to break the system down unless it is already in enough 
trouble so that there would be a good chance of its eventually breaking down by itself anyway; 
and the bigger the system grows, the more disastrous the consequences of its breakdown will be; 
so it may be that revolutionaries, by hastening the onset of the breakdown, will be reducing the 
extent of the disaster. \vspace{\baselineskip}

168. In the second place, one has to balance struggle and death against the loss of freedom and 
dignity. To many of us, freedom and dignity are more important than a long life or avoidance of 
physical pain. Besides, we all have to die some time, and it may be better to die fighting for 
survival, or for a cause, than to live a long but empty and purposeless life. \vspace{\baselineskip}

169. In the third place, it is not at all certain that survival of the system will lead to less suffering 
than breakdown of the system would. The system has already caused, and is continuing to cause, 
immense suffering all over the world. Ancient cultures, that for hundreds of years gave people a 
satisfactory relationship with each other and with their environment, have been shattered by 
contact with industrial society, and the result has been a whole catalogue of economic, 
environmental, social and psychological problems. One of the effects of the intrusion of industrial 
society has been that over much of the world traditional controls on population have been thrown 
out of balance. Hence the population explosion, with all that that implies. Then there is the 
psychological suffering that is widespread throughout the supposedly fortunate countries of the 
West (see paragraphs 44, 45). No one knows what will happen as a result of ozone depletion, the 
greenhouse effect and other environmental problems that cannot yet be foreseen. And, as nuclear 
proliferation has shown, new technology cannot be kept out of the hands of dictators and 
irresponsible Third World nations. Would you like to speculate about what Iraq or North Korea 
will do with genetic engineering? \vspace{\baselineskip}

170. “Oh!” say the technophiles, “Science is going to fix all that! We will conquer famine, 
eliminate psychological suffering, make everybody healthy and happy!” Yeah, sure. That’s what 
they said 200 years ago. The Industrial Revolution was supposed to eliminate poverty, make 
everybody happy, etc. The actual result has been quite different. The technophiles are hopelessly 
naive (or self-deceiving) in their understanding of social problems. They are unaware of (or 
choose to ignore) the fact that when large changes, even seemingly beneficial ones, are introduced 
into a society, they lead to a long sequence of other changes, most of which are impossible to 
predict (paragraph 103). The result is disruption of the society. So it is very probable that in their 
attempts to end poverty and disease, engineer docile, happy personalities and so forth, the 
technophiles will create social systems that are terribly troubled, even more so than the present 
one. For example, the scientists boast that they will end famine by creating new, genetically 
engineered food plants. But this will allow the human population to keep expanding indefinitely, 
and it is well known that crowding leads to increased stress and aggression. This is merely one 
example of the predictable problems that will arise. We emphasize that, as past experience has 
shown, technical progress will lead to other new problems that cannot be predicted in advance 
(paragraph 103). In fact, ever since the Industrial Revolution, technology has been creating new 
problems for society far more rapidly than it has been solving old ones. Thus it will take a long 
and difficult period of trial and error for the technophiles to work the bugs out of their Brave New 
World (if they every do). In the meantime there will be great suffering. So it is not at all clear that 
the survival of industrial society would involve less suffering than the breakdown of that society 
would. Technology has gotten the human race into a fix from which there is not likely to be any 
easy escape. 


\section{THE FUTURE}

\hspace{0.5cm} 171. But suppose now that industrial society does survive the next several decades and that the 
bugs do eventually get worked out of the system, so that it functions smoothly. What kind of 
system will it be? We will consider several possibilities. \vspace{\baselineskip}

172. First let us postulate that the computer scientists succeed in developing intelligent machines 
that can do all things better than human beings can do them. In that case presumably all work will 
be done by vast, highly organized systems of machines and no human effort will be 
necessary. Either of two cases might occur. The machines might be permitted to make all of their 
own decisions without human oversight, or else human control over the machines might be 
retained. \vspace{\baselineskip}

173. If the machines are permitted to make all their own decisions, we can’t make any conjectures 
as to the results, because it is impossible to guess how such machines might behave. We only 
point out that the fate of the human race would be at the mercy of the machines. It might be argued 
that the human race would never be foolish enough to hand over all power to the machines. But 
we are suggesting neither that the human race would voluntarily turn power over to the machines 
nor that the machines would willfully seize power. What we do suggest is that the human race 
might easily permit itself to drift into a position of such dependence on the machines that it would 
have no practical choice but to accept all of the machines’ decisions. As society and the problems 
that face it become more and more complex and as machines become more and more intelligent, 
people will let machines make more and more of their decisions for them, simply because machine-
made decisions will bring better results than man-made ones. Eventually a stage may be reached 
at which the decisions necessary to keep the system running will be so complex that human beings 
will be incapable of making them intelligently. At that stage the machines will be in effective 
control. People won’t be able to just turn the machine off, because they will be so dependent on 
them that turning them off would amount to suicide. \vspace{\baselineskip}

174. On the other hand it is possible that human control over the machines may be retained. In 
that case the average man may have control over certain private machines of his own, such as his 
car or his personal computer, but control over large systems of machines will be in the hands of a 
tiny elite—just as it is today, but with two differences. Due to improved techniques the elite will 
have greater control over the masses; and because human work will no longer be necessary the 
masses will be superfluous, a useless burden on the system. If the elite is ruthless they may simply 
decide to exterminate the mass of humanity. If they are humane they may use propaganda or other 
psychological or biological techniques to reduce the birth rate until the mass of humanity becomes 
extinct, leaving the world to the elite. Or, if the elite consists of soft-hearted liberals, they may 
decide to play the role of good shepherds to the rest of the human race. They will see to it that 
everyone’s physical needs are satisfied, that all children are raised under psychologically hygienic 
conditions, that everyone has a wholesome hobby to keep him busy, and that anyone who may 
become dissatisfied undergoes “treatment” to cure his “problem.” Of course, life will be so 
purposeless that people will have to be biologically or psychologically engineered either to remove 
their need for the power process or to make them “sublimate” their drive for power into some 
harmless hobby. These engineered human beings may be happy in such a society, but they most 
certainly will not be free. They will have been reduced to the status of domestic animals. \vspace{\baselineskip}

175. But suppose now that the computer scientists do not succeed in developing artificial 
intelligence, so that human work remains necessary. Even so, machines will take care of more and 
more of the simpler tasks so that there will be an increasing surplus of human workers at the lower 
levels of ability. (We see this happening already. There are many people who find it difficult or 
impossible to get work, because for intellectual or psychological reasons they cannot acquire the 
level of training necessary to make themselves useful in the present system.) On those who are 
employed, ever-increasing demands will be placed: They will need more and more training, more 
and more ability, and will have to be ever more reliable, conforming and docile, because they will 
be more and more like cells of a giant organism. Their tasks will be increasingly specialized, so 
that their work will be, in a sense, out of touch with the real world, being concentrated on one tiny 
slice of reality. The system will have to use any means that it can, whether psychological or 
biological, to engineer people to be docile, to have the abilities that the system requires and to 
“sublimate” their drive for power into some specialized task. But the statement that the people of 
such a society will have to be docile may require qualification. The society may find 
competitiveness useful, provided that ways are found of directing competitiveness into channels 
that serve the needs of the system. We can imagine a future society in which there is endless 
competition for positions of prestige and power. But no more than a very few people will ever 
reach the top, where the only real power is (see end of paragraph 163). Very repellent is a society 
in which a person can satisfy his need for power only by pushing large numbers of other people 
out of the way and depriving them of their opportunity for power. \vspace{\baselineskip}

176. One can envision scenarios that incorporate aspects of more than one of the possibilities that 
we have just discussed. For instance, it may be that machines will take over most of the work that 
is of real, practical importance, but that human beings will be kept busy by being given relatively 
unimportant work. It has been suggested, for example, that a great development of the service 
industries might provide work for human beings. Thus people would spent their time shining each 
other’s shoes, driving each other around in taxicabs, making handicrafts for one another, waiting 
on each other’s tables, etc. This seems to us a thoroughly contemptible way for the human race to 
end up, and we doubt that many people would find fulfilling lives in such pointless busy-
work. They would seek other, dangerous outlets (drugs, crime, “cults,” hate groups) unless they 
were biologically or psychologically engineered to adapt them to such a way of life. \vspace{\baselineskip}

177. Needless to say, the scenarios outlined above do not exhaust all the possibilities. They only 
indicate the kinds of outcomes that seem to us most likely. But we can envision no plausible 
scenarios that are any more palatable than the ones we’ve just described. It is overwhelmingly 
probable that if the industrial-technological system survives the next 40 to 100 years, it will by 
that time have developed certain general characteristics: Individuals (at least those of the 
“bourgeois” type, who are integrated into the system and make it run, and who therefore have all 
the power) will be more dependent than ever on large organizations; they will be more “socialized” 
than ever and their physical and mental qualities to a significant extent (possibly to a very great 
extent) will be those that are engineered into them rather than being the results of chance (or of 
God’s will, or whatever); and whatever may be left of wild nature will be reduced to remnants 
preserved for scientific study and kept under the supervision and management of scientists (hence 
it will no longer be truly wild). In the long run (say a few centuries from now) it is likely that 
neither the human race nor any other important organisms will exist as we know them today, 
because once you start modifying organisms through genetic engineering there is no reason to stop 
at any particular point, so that the modifications will probably continue until man and other 
organisms have been utterly transformed. \vspace{\baselineskip}

178. Whatever else may be the case, it is certain that technology is creating for human beings a 
new physical and social environment radically different from the spectrum of environments to 
which natural selection has adapted the human race physically and psychologically. If man is not 
adjusted to this new environment by being artificially re-engineered, then he will be adapted to it 
through a long and painful process of natural selection. The former is far more likely than the 
latter. \vspace{\baselineskip}

179. It would be better to dump the whole stinking system and take the consequences. 


\section{STRATEGY}

\hspace{0.5cm} 180. The technophiles are taking us all on an utterly reckless ride into the unknown. Many people 
understand something of what technological progress is doing to us yet take a passive attitude 
toward it because they think it is inevitable. But we (FC) don’t think it is inevitable. We think it 
can be stopped, and we will give here some indications of how to go about stopping it. \vspace{\baselineskip}

181. As we stated in paragraph 166, the two main tasks for the present are to promote social stress 
and instability in industrial society and to develop and propagate an ideology that opposes 
technology and the industrial system. When the system becomes sufficiently stressed and 
unstable, a revolution against technology may be possible. The pattern would be similar to that of 
the French and Russian Revolutions. French society and Russian society, for several decades prior 
to their respective revolutions, showed increasing signs of stress and weakness. Meanwhile, 
ideologies were being developed that offered a new world view that was quite different from the 
old one. In the Russian case, revolutionaries were actively working to undermine the ol
order. Then, when the old system was put under sufficient additional stress (by financial crisis in 
France, by military defeat in Russia) it was swept away by revolution. What we propose is 
something along the same lines. \vspace{\baselineskip} \newpage

182. It will be objected that the French and Russian Revolutions were failures. But most 
revolutions have two goals. One is to destroy an old form of society and the other is to set up the 
new form of society envisioned by the revolutionaries. The French and Russian revolutionaries 
failed (fortunately!) to create the new kind of society of which they dreamed, but they were quite 
successful in destroying the old society. We have no illusions about the feasibility of creating a 
new, ideal form of society. Our goal is only to destroy the existing form of society. \vspace{\baselineskip}

183. But an ideology, in order to gain enthusiastic support, must have a positive ideal as well as a 
negative one; it must be for something as well as against something. The positive ideal that we 
propose is Nature. That is, wild nature: those aspects of the functioning of the Earth and its living 
things that are independent of human management and free of human interference and 
control. And with wild nature we include human nature, by which we mean those aspects of the 
functioning of the human individual that are not subject to regulation by organized society but are 
products of chance, or free will, or God (depending on your religious or philosophical opinions). \vspace{\baselineskip}

184. Nature makes a perfect counter-ideal to technology for several reasons. Nature (that which 
is outside the power of the system) is the opposite of technology (which seeks to expand 
indefinitely the power of the system). Most people will agree that nature is beautiful; certainly it 
has tremendous popular appeal. The radical environmentalists already hold an ideology that exalts 
nature and opposes technology.\textsuperscript{30} It is not necessary for the sake of nature to set up some chimerical 
utopia or any new kind of social order. Nature takes care of itself: It was a spontaneous creation 
that existed long before any human society, and for countless centuries many different kinds of 
human societies coexisted with nature without doing it an excessive amount of damage. Only with 
the Industrial Revolution did the effect of human society on nature become really devastating. To 
relieve the pressure on nature it is not necessary to create a special kind of social system, it is only 
necessary to get rid of industrial society. Granted, this will not solve all problems. Industrial 
society has already done tremendous damage to nature and it will take a very long time for the 
scars to heal. Besides, even preindustrial societies can do significant damage to 
nature. Nevertheless, getting rid of industrial society will accomplish a great deal. It will relieve
the worst of the pressure on nature so that the scars can begin to heal. It will remove the capacity 
of organized society to keep increasing its control over nature (including human nature). Whatever 
kind of society may exist after the demise of the industrial system, it is certain that most people 
will live close to nature, because in the absence of advanced technology there is no other way that 
people can live. To feed themselves they must be peasants or herdsmen or fishermen or hunters, 
etc. And, generally speaking, local autonomy should tend to increase, because lack of advanced 
technology and rapid communications will limit the capacity of governments or other large 
organizations to control local communities. \vspace{\baselineskip}

185. As for the negative consequences of eliminating industrial society—well, you can’t eat your 
cake and have it too. To gain one thing you have to sacrifice another. \vspace{\baselineskip}

186. Most people hate psychological conflict. For this reason they avoid doing any serious 
thinking about difficult social issues, and they like to have such issues presented to them in simple, 
black-and-white terms: this is all good and that is all bad. The revolutionary ideology should 
therefore be developed on two levels. \vspace{\baselineskip}

187. On the more sophisticated level the ideology should address itself to people who are 
intelligent, thoughtful and rational. The object should be to create a core of people who will be 
opposed to the industrial system on a rational, thought-out basis, with full appreciation of the 
problems and ambiguities involved, and of the price that has to be paid for getting rid of the 
system. It is particularly important to attract people of this type, as they are capable people and 
will be instrumental in influencing others. These people should be addressed on as rational a level 
as possible. Facts should never intentionally be distorted and intemperate language should be 
avoided. This does not mean that no appeal can be made to the emotions, but in making such 
appeal care should be taken to avoid misrepresenting the truth or doing anything else that would 
destroy the intellectual respectability of the ideology. \vspace{\baselineskip}

188. On a second level, the ideology should be propagated in a simplified form that will enable 
the unthinking majority to see the conflict of technology vs. nature in unambiguous terms. But 
even on this second level the ideology should not be expressed in language that is so cheap, 
intemperate or irrational that it alienates people of the thoughtful and rational type. Cheap, 
intemperate propaganda sometimes achieves impressive short-term gains, but it will be more 
advantageous in the long run to keep the loyalty of a small number of intelligently committed 
people than to arouse the passions of an unthinking, fickle mob who will change their attitude as 
soon as someone comes along with a better propaganda gimmick. However, propaganda of the 
rabble-rousing type may be necessary when the system is nearing the point of collapse and there 
is a final struggle between rival ideologies to determine which will become dominant when the old 
world-view goes under. \vspace{\baselineskip}

189. Prior to that final struggle, the revolutionaries should not expect to have a majority of people 
on their side. History is made by active, determined minorities, not by the majority, which seldom 
has a clear and consistent idea of what it really wants. Until the time comes for the final push 
toward revolution,\textsuperscript{31} the task of revolutionaries will be less to win the shallow support of the 
majority than to build a small core of deeply committed people. As for the majority, it will be 
enough to make them aware of the existence of the new ideology and remind them of it frequently; 
though of course it will be desirable to get majority support to the extent that this can be done 
without weakening the core of seriously committed people. \vspace{\baselineskip} \newpage

190. Any kind of social conflict helps to destabilize the system, but one should be careful about 
what kind of conflict one encourages. The line of conflict should be drawn between the mass of 
the people and the power-holding elite of industrial society (politicians, scientists, upper-level 
business executives, government officials, etc.). It should not be drawn between the 
revolutionaries and the mass of the people. For example, it would be bad strategy for the 
revolutionaries to condemn Americans for their habits of consumption. Instead, the average 
American should be portrayed as a victim of the advertising and marketing industry, which has 
suckered him into buying a lot of junk that he doesn’t need and that is very poor compensation for 
his lost freedom. Either approach is consistent with the facts. It is merely a matter of attitude 
whether you blame the advertising industry for manipulating the public or blame the public for 
allowing itself to be manipulated. As a matter of strategy one should generally avoid blaming the 
public. \vspace{\baselineskip}

191. One should think twice before encouraging any other social conflict than that between the 
power-holding elite (which wields technology) and the general public (over which technology 
exerts its power). For one thing, other conflicts tend to distract attention from the important 
conflicts (between power-elite and ordinary people, between technology and nature); for another 
thing, other conflicts may actually tend to encourage technologization, because each side in such 
a conflict wants to use technological power to gain advantages over its adversary. This is clearly 
seen in rivalries between nations. It also appears in ethnic conflicts within nations. For example, 
in America many black leaders are anxious to gain power for African Americans by placing black 
individuals in the technological power-elite. They want there to be many black government 
officials, scientists, corporation executives and so forth. In this way they are helping to absorb the 
African American subculture into the technological system. Generally speaking, one should 
encourage only those social conflicts that can be fitted into the framework of the conflicts of 
power-elite vs. ordinary people, technology vs. nature. \vspace{\baselineskip}

192. But the way to discourage ethnic conflict is not through militant advocacy of minority rights 
(see paragraphs 21, 29). Instead, the revolutionaries should emphasize that although minorities do 
suffer more or less disadvantage, this disadvantage is of peripheral significance. Our real enemy 
is the industrial-technological system, and in the struggle against the system, ethnic distinctions 
are of no importance. \vspace{\baselineskip}

193. The kind of revolution we have in mind will not necessarily involve an armed uprising 
against any government. It may or may not involve physical violence, but it will not be a political 
revolution. Its focus will be on technology and economics, not politics.\textsuperscript{32} \vspace{\baselineskip}

194. Probably the revolutionaries should even avoid assuming political power, whether by legal 
or illegal means, until the industrial system is stressed to the danger point and has proved itself to 
be a failure in the eyes of most people. Suppose for example that some “green” party should win 
control of the United States Congress in an election. In order to avoid betraying or watering down 
their own ideology they would have to take vigorous measures to turn economic growth into
economic shrinkage. To the average man the results would appear disastrous: There would be 
massive unemployment, shortages of commodities, etc. Even if the grosser ill effects could be 
avoided through superhumanly skillful management, still people would have to begin giving up 
the luxuries to which they have become addicted. Dissatisfaction would grow, the “green” party 
would be voted out of office and the revolutionaries would have suffered a severe setback. For 
this reason the revolutionaries should not try to acquire political power until the system has gotten 
itself into such a mess that any hardships will be seen as resulting from the failures of the industrial 
system itself and not from the policies of the revolutionaries. The revolution against technology 
will probably have to be a revolution by outsiders, a revolution from below and not from above. \vspace{\baselineskip}

195. The revolution must be international and worldwide. It cannot be carried out on a nation-
by-nation basis. Whenever it is suggested that the United States, for example, should cut back on 
technological progress or economic growth, people get hysterical and start screaming that if we 
fall behind in technology the Japanese will get ahead of us. Holy robots! The world will fly off its 
orbit if the Japanese ever sell more cars than we do! (Nationalism is a great promoter of 
technology.) More reasonably, it is argued that if the relatively democratic nations of the world 
fall behind in technology while nasty, dictatorial nations like China, Vietnam and North Korea 
continue to progress, eventually the dictators may come to dominate the world. That is why the 
industrial system should be attacked in all nations simultaneously, to the extent that this may be 
possible. True, there is no assurance that the industrial system can be destroyed at approximately 
the same time all over the world, and it is even conceivable that the attempt to overthrow the 
system could lead instead to the domination of the system by dictators. That is a risk that has to 
be taken. And it is worth taking, since the difference between a “democratic” industrial system 
and one controlled by dictators is small compared with the difference between an industrial system 
and a non-industrial one.\textsuperscript{33} It might even be argued that an industrial system controlled by dictators 
would be preferable, because dictator-controlled systems usually have proved inefficient, hence 
they are presumably more likely to break down. Look at Cuba. \vspace{\baselineskip}

196. Revolutionaries might consider favoring measures that tend to bind the world economy into 
a unified whole. Free trade agreements like NAFTA and GATT are probably harmful to the 
environment in the short run, but in the long run they may perhaps be advantageous because they 
foster economic interdependence between nations. It will be easier to destroy the industrial system 
on a worldwide basis if the world economy is so unified that its breakdown in any one major nation 
will lead to its breakdown in all industrialized nations. \vspace{\baselineskip}

197. Some people take the line that modern man has too much power, too much control over 
nature; they argue for a more passive attitude on the part of the human race. At best these people 
are expressing themselves unclearly, because they fail to distinguish between power for large 
organizations and power for individuals and small groups. It is a mistake to argue for 
powerlessness and passivity, because people need power. Modern man as a collective entity—
that is, the industrial system—has immense power over nature, and we (FC) regard this as evil. But 
modern individuals and small groups of individuals have far less power than primitive man ever 
did. Generally speaking, the vast power of “modern man” over nature is exercised not by 
individuals or small groups but by large organizations. To the extent that the average modern 
individual can wield the power of technology, he is permitted to do so only within narrow limits 
and only under the supervision and control of the system. (You need a license for everything and 
with the license come rules and regulations.) The individual has only those technological powers 
with which the system chooses to provide him. His personal power over nature is slight. \vspace{\baselineskip}

198. Primitive individuals and small groups actually had considerable power over nature; or
maybe it would be better to say power within nature. When primitive man needed food he knew 
how to find and prepare edible roots, how to track game and take it with homemade weapons. He 
knew how to protect himself from heat cold, rain, dangerous animals, etc. But primitive man did 
relatively little damage to nature because the collective power of primitive society was negligible 
compared to the collective power of industrial society. \vspace{\baselineskip}

199. Instead of arguing for powerlessness and passivity, one should argue that the power of the 
industrial system should be broken, and that this will greatly increase the power and freedom of 
individuals and small groups. \vspace{\baselineskip}

200. Until the industrial system has been thoroughly wrecked, the destruction of that system must 
be the revolutionaries’ only goal. Other goals would distract attention and energy from the main 
goal. More importantly if the revolutionaries permit themselves to have any other goal than the 
destruction of technology, they will be tempted to use technology as a tool for reaching that other 
goal. If they give in to that temptation, they will fall right back into the technological trap, because 
modern technology is a unified, tightly organized system, so that, in order to retain some 
technology, one finds oneself obliged to retain most technology, hence one ends up sacrificing 
only token amounts of technology. \vspace{\baselineskip}

201. Suppose for example that the revolutionaries took “social justice” as a goal. Human nature 
being what it is, social justice would not come about spontaneously; it would have to be 
enforced. In order to enforce it the revolutionaries would have to retain central organization and 
control. For that they would need rapid long-distance transportation and communication, and 
therefore all the technology needed to support the transportation and communication systems. To 
feed and clothe poor people they would have to use agricultural and manufacturing 
technology. And so forth. So that the attempt to insure social justice would force them to retain 
most parts of the technological system. Not that we have anything against social justice, but it 
must not be allowed to interfere with the effort to get rid of the technological system. \vspace{\baselineskip}

202. It would be hopeless for revolutionaries to try to attack the system without using some 
modern technology. If nothing else they must use the communications media to spread their 
message. But they should use modern technology for only one purpose: to attack the technological 
system. \vspace{\baselineskip}

203. Imagine an alcoholic sitting with a barrel of wine in front of him. Suppose he starts saying 
to himself, “Wine isn’t bad for you if used in moderation. Why, they say small amounts of wine 
are even good for you! It won’t do me any harm if I take just one little drink....” Well you know 
what is going to happen. Never forget that the human race with technology is just like an alcoholic 
with a barrel of wine. \vspace{\baselineskip}

204. Revolutionaries should have as many children as they can. There is strong scientific 
evidence that social attitudes are to a significant extent inherited. No one suggests that a social 
attitude is a direct outcome of a person’s genetic constitution, but it appears that personality traits 
are partly inherited and that certain personality traits tend, within the context of our society, to 
make a person more likely to hold this or that social attitude. Objections to these findings have 
been raised, but the objections are feeble and seem to be ideologically motivated. In any event, no 
one denies that children tend on the average to hold social attitudes similar to those of their 
parents. From our point of view it doesn’t matter all that much whether the attitudes are passed 
on genetically or through childhood training. In either case they are passed on. \vspace{\baselineskip}

205. The trouble is that many of the people who are inclined to rebel against the industrial system 
are also concerned about the population problems, hence they are apt to have few or no children. In 
this way they may be handing the world over to the sort of people who support or at least accept 
the industrial system. To ensure the strength of the next generation of revolutionaries the present 
generation should reproduce itself abundantly. In doing so they will be worsening the population 
problem only slightly. And the important problem is to get rid of the industrial system, because 
once the industrial system is gone the world’s population necessarily will decrease (see paragraph 
167); whereas, if the industrial system survives, it will continue developing new techniques of food 
production that may enable the world’s population to keep increasing almost indefinitely. \vspace{\baselineskip}

206. With regard to revolutionary strategy, the only points on which we absolutely insist are that 
the single overriding goal must be the elimination of modern technology, and that no other goal 
can be allowed to compete with this one. For the rest, revolutionaries should take an empirical 
approach. If experience indicates that some of the recommendations made in the foregoing 
paragraphs are not going to give good results, then those recommendations should be discarded. 


\section{TWO KINDS OF TECHNOLOGY}

\hspace{0.5cm} 207. An argument likely to be raised against our proposed revolution is that it is bound to fail, 
because (it is claimed) throughout history technology has always progressed, never regressed, 
hence technological regression is impossible. But this claim is false. \vspace{\baselineskip}

208. We distinguish between two kinds of technology, which we will call small-scale technology 
and organization-dependent technology. Small-scale technology is technology that can be used 
by small-scale communities without outside assistance. Organization-dependent technology is 
technology that depends on large-scale social organization. We are aware of no significant cases 
of regression in small-scale technology. But organization-dependent technology does regress 
when the social organization on which it depends breaks down. Example: When the Roman 
Empire fell apart the Romans’ small-scale technology survived because any clever village 
craftsman could build, for instance, a water wheel, any skilled smith could make steel by Roman 
methods, and so forth. But the Romans’ organization-dependent technology did regress. Their 
aqueducts fell into disrepair and were never rebuilt. Their techniques of road construction were 
lost. The Roman system of urban sanitation was forgotten, so that not until rather recent times did 
the sanitation of European cities equal that of Ancient Rome. \vspace{\baselineskip}

209. The reason why technology has seemed always to progress is that, until perhaps a century or 
two before the Industrial Revolution, most technology was small-scale technology. But most of 
the technology developed since the Industrial Revolution is organization-dependent 
technology. Take the refrigerator for example. Without factory-made parts or the facilities of a 
postindustrial machine shop it would be virtually impossible for a handful of local craftsmen to 
build a refrigerator. If by some miracle they did succeed in building one it would be useless to 
them without a reliable source of electric power. So they would have to dam a stream and build a 
generator. Generators require large amounts of copper wire. Imagine trying to make that wire 
without modern machinery. And where would they get a gas suitable for refrigeration? It would 
be much easier to build an icehouse or preserve food by drying or picking, as was done before the 
invention of the refrigerator. \vspace{\baselineskip}

210. So it is clear that if the industrial system were once thoroughly broken down, refrigeration 
technology would quickly be lost. The same is true of other organization-dependent 
technology. And once this technology had been lost for a generation or so it would take centuries 
to rebuild it, just as it took centuries to build it the first time around. Surviving technical books 
would be few and scattered. An industrial society, if built from scratch without outside help, can 
only be built in a series of stages: You need tools to make tools to make tools to make tools... A 
long process of economic development and progress in social organization is required. And, even 
in the absence of an ideology opposed to technology, there is no reason to believe that anyone 
would be interested in rebuilding industrial society. The enthusiasm for “progress” is a 
phenomenon peculiar to the modern form of society, and it seems not to have existed prior to the 
17th century or thereabouts. \vspace{\baselineskip}

211. In the late Middle Ages there were four main civilizations that were about equally 
“advanced”: Europe, the Islamic world, India, and the Far East (China, Japan, Korea). Three of 
those civilizations remained more or less stable, and only Europe became dynamic. No one knows 
why Europe became dynamic at that time; historians have their theories but these are only 
speculation. At any rate, it is clear that rapid development toward a technological form of society 
occurs only under special conditions. So there is no reason to assume that a long-lasting 
technological regression cannot be brought about. \vspace{\baselineskip}

212. Would society eventually develop again toward an industrial-technological form? Maybe, 
but there is no use in worrying about it, since we can’t predict or control events 500 or 1,000 years 
in the future. Those problems must be dealt with by the people who will live at that time. 


\section{THE DANGER OF LEFTISM}

\hspace{0.5cm} 213. Because of their need for rebellion and for membership in a movement, leftists or persons of 
similar psychological type often are unattracted to a rebellious or activist movement whose goals 
and membership are not initially leftist. The resulting influx of leftish types can easily turn a non-
leftist movement into a leftist one, so that leftist goals replace or distort the original goals of the 
movement. \vspace{\baselineskip} \newpage

214. To avoid this, a movement that exalts nature and opposes technology must take a resolutely 
anti-leftist stance and must avoid all collaboration with leftists. Leftism is in the long run 
inconsistent with wild nature, with human freedom and with the elimination of modern 
technology. Leftism is collectivist; it seeks to bind together the entire world (both nature and the 
human race) into a unified whole. But this implies management of nature and of human life by 
organized society, and it requires advanced technology. You can’t have a united world without 
rapid transportation and communication, you can’t make all people love one another without 
sophisticated psychological techniques, you can’t have a “planned society” without the necessary 
technological base. Above all, leftism is driven by the need for power, and the leftist seeks power 
on a collective basis, through identification with a mass movement or an organization. Leftism is 
unlikely ever to give up technology, because technology is too valuable a source of collective 
power. \vspace{\baselineskip}

215. The anarchist\textsuperscript{34} too seeks power, but he seeks it on an individual or small-group basis; he 
wants individuals and small groups to be able to control the circumstances of their own lives. He 
opposes technology because it makes small groups dependent on large organizations. \vspace{\baselineskip}

216. Some leftists may seem to oppose technology, but they will oppose it only so long as they 
are outsiders and the technological system is controlled by non-leftists. If leftism ever becomes 
dominant in society, so that the technological system becomes a tool in the hands of leftists, they 
will enthusiastically use it and promote its growth. In doing this they will be repeating a pattern 
that leftism has shown again and again in the past. When the Bolsheviks in Russia were outsiders, 
they vigorously opposed censorship and the secret police, they advocated self-determination for 
ethnic minorities, and so forth; but as soon as they came into power themselves, they imposed a 
tighter censorship and created a more ruthless secret police than any that had existed under the 
tsars, and they oppressed ethnic minorities at least as much as the tsars had done. In the United 
States, a couple of decades ago when leftists were a minority in our universities, leftist professors 
were vigorous proponents of academic freedom, but today, in those of our universities where 
leftists have become dominant, they have shown themselves ready to take away from everyone 
else’s academic freedom. (This is “political correctness.”) The same will happen with leftists and 
technology: They will use it to oppress everyone else if they ever get it under their own control. \vspace{\baselineskip}

217. In earlier revolutions, leftists of the most power-hungry type, repeatedly, have first 
cooperated with non-leftist revolutionaries, as well as with leftists of a more libertarian inclination, 
and later have double-crossed them to seize power for themselves. Robespierre did this in the 
French Revolution, the Bolsheviks did it in the Russian Revolution, the communists did it in Spain 
in 1938 and Castro and his followers did it in Cuba. Given the past history of leftism, it would be 
utterly foolish for non-leftist revolutionaries today to collaborate with leftists. \vspace{\baselineskip}

218. Various thinkers have pointed out that leftism is a kind of religion. Leftism is not a religion 
in the strict sense because leftist doctrine does not postulate the existence of any supernatural 
being. But, for the leftist, leftism plays a psychological role much like that which religion plays 
for some people. The leftist needs to believe in leftism; it plays a vital role in his psychological 
economy. His beliefs are not easily modified by logic or facts. He has a deep conviction that 
leftism is morally Right with a capital R, and that he has not only a right but a duty to impose 
leftist morality on everyone. (However, many of the people we are referring to as “leftists” do not 
think of themselves as leftists and would not describe their system of beliefs as leftism. We use 
the term “leftism” because we don’t know of any better words to designate the spectrum of related 
creeds that includes the feminist, gay rights, political correctness, etc., movements, and because 
these movements have a strong affinity with the old left. See paragraphs 227-230.) \vspace{\baselineskip}

219. Leftism is a totalitarian force. Wherever leftism is in a position of power it tends to invade 
every private corner and force every thought into a leftist mold. In part this is because of the quasi-
religious character of leftism: everything contrary to leftist beliefs represents Sin. More 
importantly, leftism is a totalitarian force because of the leftists’ drive for power. The leftist seeks 
to satisfy his need for power through identification with a social movement and he tries to go 
through the power process by helping to pursue and attain the goals of the movement (see 
paragraph 83). But no matter how far the movement has gone in attaining its goals the leftist is 
never satisfied, because his activism is a surrogate activity (see paragraph 41). That is, the leftist’s 
real motive is not to attain the ostensible goals of leftism; in reality he is motivated by the sense of 
power he gets from struggling for and then reaching a social goal.\textsuperscript{35} Consequently the leftist is 
never satisfied with the goals he has already attained; his need for the power process leads him 
always to pursue some new goal. The leftist wants equal opportunities for minorities. When that 
is attained he insists on statistical equality of achievement by minorities. And as long as anyone 
harbors in some corner of his mind a negative attitude toward some minority, the leftist has to re-
educated him. And ethnic minorities are not enough; no one can be allowed to have a negative 
attitude toward homosexuals, disabled people, fat people, old people, ugly people, and on and on 
and on. It’s not enough that the public should be informed about the hazards of smoking; a warning 
has to be stamped on every package of cigarettes. Then cigarette advertising has to be restricted 
if not banned. The activists will never be satisfied until tobacco is outlawed, and after that it will 
be alcohol, then junk food, etc. Activists have fought gross child abuse, which is reasonable. But 
now they want to stop all spanking. When they have done that they will want to ban something 
else they consider unwholesome, then another thing and then another. They will never be satisfied 
until they have complete control over all child rearing practices. And then they will move on to 
another cause. \vspace{\baselineskip}

220. Suppose you asked leftists to make a list of all the things that were wrong with society, and 
then suppose you instituted every social change that they demanded. It is safe to say that within a 
couple of years the majority of leftists would find something new to complain about, some new 
social “evil” to correct; because, once again, the leftist is motivated less by distress at society’s ills 
than by the need to satisfy his drive for power by imposing his solutions on society. \vspace{\baselineskip}

221. Because of the restrictions placed on their thoughts and behavior by their high level of 
socialization, many leftists of the over-socialized type cannot pursue power in the ways that other 
people do. For them the drive for power has only one morally acceptable outlet, and that is in the 
struggle to impose their morality on everyone. \vspace{\baselineskip}

222. Leftists, especially those of the oversocialized type, are True Believers in the sense of Eric 
Hoffer’s book, The True Believer. But not all True Believers are of the same psychological type 
as leftists. Presumably a true-believing nazi, for instance, is very different psychologically from a 
true-believing leftist. Because of their capacity for single-minded devotion to a cause, True 
Believers are a useful, perhaps a necessary, ingredient of any revolutionary movement. This 
presents a problem with which we must admit we don’t know how to deal. We aren’t sure how to 
harness the energies of the True Believer to a revolution against technology. At present all we can 
say is that no True Believer will make a safe recruit to the revolution unless his commitment is 
exclusively to the destruction of technology. If he is committed also to another ideal, he may want 
to use technology as a tool for pursuing that other ideal (see paragraphs 200, 201). \vspace{\baselineskip}

223. Some readers may say, “This stuff about leftism is a lot of crap. I know John and Jane who 
are leftish types and they don’t have all these totalitarian tendencies.” It’s quite true that many 
leftists, possibly even a numerical majority, are decent people who sincerely believe in tolerating 
others’ values (up to a point) and wouldn’t want to use high-handed methods to reach their social 
goals. Our remarks about leftism are not meant to apply to every individual leftist but to describe 
the general character of leftism as a movement. And the general character of a movement is not 
necessarily determined by the numerical proportions of the various kinds of people involved in the 
movement. \vspace{\baselineskip}

224. The people who rise to positions of power in leftist movements tend to be leftists of the most 
power-hungry type, because power-hungry people are those who strive hardest to get into positions 
of power. Once the power-hungry types have captured control of the movement, there are many 
leftists of a gentler breed who inwardly disapprove of many of the actions of the leaders, but cannot 
bring themselves to oppose them. They need their faith in the movement, and because they cannot 
give up this faith they go along with the leaders. True, some leftists do have the guts to oppose the 
totalitarian tendencies that emerge, but they generally lose, because the power-hungry types are 
better organized, are more ruthless and Machiavellian and have taken care to build themselves a 
strong power base. \vspace{\baselineskip}

225. These phenomena appeared clearly in Russia and other countries that were taken over by 
leftists. Similarly, before the breakdown of communism in the, USSR, leftish types in the West 
would seldom criticize that country. If prodded they would admit that the USSR did many wrong 
things, but then they would try to find excuses for the communists and begin talking about the 
faults of the West. They always opposed Western military resistance to communist 
aggression. Leftish types all over the world vigorously protested the U.S. military action in 
Vietnam, but when the USSR invaded Afghanistan they did nothing. Not that they approved of 
the Soviet actions; but because of their leftist faith, they just couldn’t bear to put themselves in 
opposition to communism. Today, in those of our universities where “political correctness” has 
become dominant, there are probably many leftish types who privately disapprove of the 
suppression of academic freedom, but they go along with it anyway. \vspace{\baselineskip}

226. Thus the fact that many individual leftists are personally mild and fairly tolerant people by 
no means prevents leftism as a whole form having a totalitarian tendency. \vspace{\baselineskip} \newpage

227. Our discussion of leftism has a serious weakness. It is still far from clear what we mean by 
the word “leftist.” There doesn’t seem to be much we can do about this. Today leftism is 
fragmented into a whole spectrum of activist movements. Yet not all activist movements are 
leftist, and some activist movements (e.g., radical environmentalism) seem to include both 
personalities of the leftist type and personalities of thoroughly un-leftist types who ought to know 
better than to collaborate with leftists. Varieties of leftists fade out gradually into varieties of non-
leftists and we ourselves would often be hard-pressed to decide whether a given individual is or is 
not a leftist. To the extent that it is defined at all, our conception of leftism is defined by the 
discussion of it that we have given in this article, and we can only advise the reader to use his own 
judgment in deciding who is a leftist. \vspace{\baselineskip}

228. But it will be helpful to list some criteria for diagnosing leftism. These criteria cannot be 
applied in a cut and dried manner. Some individuals may meet some of the criteria without being 
leftists, some leftists may not meet any of the criteria. Again, you just have to use your judgment. \vspace{\baselineskip}

229. The leftist is oriented toward large-scale collectivism. He emphasizes the duty of the 
individual to serve society and the duty of society to take care of the individual. He has a negative 
attitude toward individualism. He often takes a moralistic tone. He tends to be for gun control, 
for sex education and other psychologically “enlightened” educational methods, for social 
planning, for affirmative action, for multiculturalism. He tends to identify with victims. He tends 
to be against competition and against violence, but he often finds excuses for those leftists who do 
commit violence. He is fond of using the common catch-phrases of the left, like “racism,” 
“sexism,” “homophobia,” “capitalism,” “imperialism,” “neocolonialism,” “genocide,” “social 
change,” “social justice,” “social responsibility.” Maybe the best diagnostic trait of the leftist is 
his tendency to sympathize with the following movements: feminism, gay rights, ethnic rights, 
disability rights, animal rights, political correctness. Anyone who strongly sympathizes with all 
of these movements is almost certainly a leftist.\textsuperscript{36} \vspace{\baselineskip}

230. The more dangerous leftists, that is, those who are most power-hungry, are often 
characterized by arrogance or by a dogmatic approach to ideology. However, the most dangerous 
leftists of all may be certain oversocialized types who avoid irritating displays of aggressiveness 
and refrain from advertising their leftism, but work quietly and unobtrusively to promote 
collectivist values, “enlightened” psychological techniques for socializing children, dependence 
of the individual on the system, and so forth. These crypto-leftists (as we may call them) 
approximate certain bourgeois types as far as practical action is concerned, but differ from them 
in psychology, ideology and motivation. The ordinary bourgeois tries to bring people under 
control of the system in order to protect his way of life, or he does so simply because his attitudes 
are conventional. The crypto-leftist tries to bring people under control of the system because he 
is a True Believer in a collectivistic ideology. The crypto-leftist is differentiated from the average 
leftist of the oversocialized type by the fact that his rebellious impulse is weaker and he is more 
securely socialized. He is differentiated from the ordinary well-socialized bourgeois by the fact 
that there is some deep lack within him that makes it necessary for him to devote himself to a cause 
and immerse himself in a collectivity. And maybe his (well-sublimated) drive for power is 
stronger than that of the average bourgeois. 


\section{FINAL NOTE}

\hspace{0.5cm} 231. Throughout this article we’ve made imprecise statements and statements that ought to have 
had all sorts of qualifications and reservations attached to them; and some of our statements may 
be flatly false. Lack of sufficient information and the need for brevity made it impossible for us 
to formulate our assertions more precisely or add all the necessary qualifications. And of course 
in a discussion of this kind one must rely heavily on intuitive judgment, and that can sometimes 
be wrong. So we don’t claim that this article expresses more than a crude approximation to the 
truth. \vspace{\baselineskip} 

232. All the same, we are reasonably confident that the general outlines of the picture we have 
painted here are roughly correct. Just one possible weak point needs to be mentioned. We have 

portrayed leftism in its modern form as a phenomenon peculiar to our time and as a symptom of 
the disruption of the power process. But we might possibly be wrong about this. Oversocialized 
types who try to satisfy their drive for power by imposing their morality on everyone have certainly 
been around for a long time. But we think that the decisive role played by feelings of inferiority, 
low self-esteem, powerlessness, identification with victims by people who are not themselves
victims, is a peculiarity of modern leftism. Identification with victims by people not themselves 
victims can be seen to some extent in 19th century leftism and early Christianity, but as far as we 
can make out, symptoms of low self-esteem, etc., were not nearly so evident in these movements, 
or in any other movements, as they are in modern leftism. But we are not in a position to assert 
confidently that no such movements have existed prior to modern leftism. This is a significant 
question to which historians ought to give their attention. 

\afterpage{\null\newpage}
\newpage

\section{NOTES}

\hspace{0.5cm} \textsuperscript{1} (Paragraph 19) We are not asserting that all, or even most, bullies and ruthless competitors 
suffer from feelings of inferiority. \vspace{\baselineskip} 

\textsuperscript{2} (Paragraph 25) During the Victorian period many oversocialized people suffered from serious 
psychological problems as a result of repressing or trying to repress their sexual feelings. Freud 
apparently based his theories on people of this type. Today the focus of socialization has shifted 
from sex to aggression. \vspace{\baselineskip}

\textsuperscript{3} (Paragraph 27) Not necessarily including specialists in engineering or the “hard” sciences. \vspace{\baselineskip}

\textsuperscript{4} (Paragraph 28) There are many individuals of the middle and upper classes who resist some of 
these values, but usually their resistance is more or less covert. Such resistance appears in the 
mass media only to a very limited extent. The main thrust of propaganda in our society is in 
favor of the stated values. The main reason why these values have become, so to speak, the 
official values of our society is that they are useful to the industrial system. Violence is 
discouraged because it disrupts the functioning of the system. Racism is discouraged because 
ethnic conflicts also disrupt the system, and discrimination wastes the talents of minority-group 
members who could be useful to the system. Poverty must be “cured” because the underclass 
causes problems for the system and contact with the underclass lowers the morale of the other 
classes. Women are encouraged to have careers because their talents are useful to the system 
and, more importantly, because by having regular jobs women become better integrated into the 
system and tied directly to it rather than to their families. This helps to weaken family 
solidarity. (The leaders of the system say they want to strengthen the family, but they really 
mean is that they want the family to serve as an effective tool for socializing children in accord 
with the needs of the system. We argue in paragraphs 51, 52 that the system cannot afford to let 
the family or other small-scale social groups be strong or autonomous.) \vspace{\baselineskip}

\textsuperscript{5} (Paragraph 42) It may be argued that the majority of people don’t want to make their own 
decisions but want leaders to do their thinking for them. There is an element of truth in 
this. People like to make their own decisions in small matters, but making decisions on difficult, 
fundamental questions requires facing up to psychological conflict, and most people hate 
psychological conflict. Hence they tend to lean on others in making difficult decisions. But it 
does not follow that they like to have decisions imposed upon them without having any 
opportunity to influence those decisions. The majority of people are natural followers, not 
leaders, but they like to have direct personal access to their leaders, they want to be able to 
influence the leaders and participate to some extent in making even the difficult decisions. At 
least to that degree they need autonomy. \vspace{\baselineskip}

\textsuperscript{6} (Paragraph 44) Some of the symptoms listed are similar to those shown by caged animals. To 
explain how these symptoms arise from deprivation with respect to the power process: common-
sense understanding of human nature tells one that lack of goals whose attainment requires effort 
leads to boredom and that boredom, long continued, often leads eventually to 
depression. Failure to attain goals leads to frustration and lowering of self-esteem. Frustration 
leads to anger, anger to aggression, often in the form of spouse or child abuse. It has been shown 
that long-continued frustration commonly leads to depression and that depression tends to cause 
guilt, sleep disorders, eating disorders and bad feelings about oneself. Those who are tending 
toward depression seek pleasure as an antidote; hence insatiable hedonism and excessive sex, 
with perversions as a means of getting new kicks. Boredom too tends to cause excessive 
pleasure-seeking since, lacking other goals, people often use pleasure as a goal. \vspace{\baselineskip}

\textsuperscript{7} (Paragraph 52) A partial exception may be made for a few passive, inward-looking groups, such 
as the Amish, which have little effect on the wider society. Apart from these, some genuine 
small-scale communities do exist in America today. For instance, youth gangs and “cults.” 
Everyone regards them as dangerous, and so they are, because the members of these groups are 
loyal primarily to one another rather than to the system, hence the system cannot control 
them. Or take the gypsies. The gypsies commonly get away with theft and fraud because their 
loyalties are such that they can always get other gypsies to give testimony that “proves” their 
innocence. Obviously the system would be in serious trouble if too many people belonged to 
such groups. Some of the early-20th century Chinese thinkers who were concerned with 
modernizing China recognized the necessity breaking down small-scale social groups such as the 
family: “(According to Sun Yat-sen) the Chinese people needed a new surge of patriotism, which 
would lead to a transfer of loyalty from the family to the state.... (According to Li Huang) 
traditional attachments, particularly to the family had to be abandoned if nationalism were to 
develop in China.” (Chester C. Tan, “Chinese Political Thought in the Twentieth Century,” page 
125, page 297.) \vspace{\baselineskip}

\textsuperscript{8} (Paragraph 56) Yes, we know that 19th century America had its problems, and serious ones, but 
for the sake of brevity we have to express ourselves in simplified terms. \vspace{\baselineskip} \newpage

\textsuperscript{9} (Paragraph 61) We leave aside the “underclass.” We are speaking of the mainstream. \vspace{\baselineskip}

\textsuperscript{10} (Paragraph 62) Some social scientists, educators, “mental health” professionals and the like are 
doing their best to push the social drives into group 1 by trying to see to it that everyone has a 
satisfactory social life. \vspace{\baselineskip}

\textsuperscript{11} (Paragraphs 63, 82) Is the drive for endless material acquisition really an artificial creation of 
the advertising and marketing industry? Certainly there is no innate human drive for material 
acquisition. There have been many cultures in which people have desired little material wealth 
beyond what was necessary to satisfy their basic physical needs (Australian aborigines, 
traditional Mexican peasant culture, some African cultures). On the other hand there have also 
been many pre-industrial cultures in which material acquisition has played an important role. So 
we can’t claim that today’s acquisition-oriented culture is exclusively a creation of the 
advertising and marketing industry. But it is clear that the advertising and marketing industry 
has had an important part in creating that culture. The big corporations that spend millions on 
advertising wouldn’t be spending that kind of money without solid proof that they were getting it 
back in increased sales. One member of FC met a sales manager a couple of years ago who was 
frank enough to tell him, “Our job is to make people buy things they don’t want and don’t need.” 
He then described how an untrained novice could present people with the facts about a product, 
and make no sales at all, while a trained and experienced professional salesman would make lots 
of sales to the same people. This shows that people are manipulated into buying things they 
don’t really want. \vspace{\baselineskip}

\textsuperscript{12} (Paragraph 64) The problem of purposelessness seems to have become less serious during the 
last 15 years or so, because people now feel less secure physically and economically than they 
did earlier, and the need for security provides them with a goal. But purposelessness has been 
replaced by frustration over the difficulty of attaining security. We emphasize the problem of 
purposelessness because the liberals and leftists would wish to solve our social problems by 
having society guarantee everyone’s security; but if that could be done it would only bring back 
the problem of purposelessness. The real issue is not whether society provides well or poorly for 
people’s security; the trouble is that people are dependent on the system for their security rather 
than having it in their own hands. This, by the way, is part of the reason why some people get 
worked up about the right to bear arms; possession of a gun puts that aspect of their security in 
their own hands. \vspace{\baselineskip} \newpage

\textsuperscript{13} (Paragraph 66) Conservatives’ efforts to decrease the amount of government regulation are of 
little benefit to the average man. For one thing, only a fraction of the regulations can be 
eliminated because most regulations are necessary. For another thing, most of the deregulation 
affects business rather than the average individual, so that its main effect is to take power from 
the government and give it to private corporations. What this means for the average man is that 
government interference in his life is replaced by interference from big corporations, which may 
be permitted, for example, to dump more chemicals that get into his water supply and give him 
cancer. The conservatives are just taking the average man for a sucker, exploiting his resentment 
of Big Government to promote the power of Big Business. \vspace{\baselineskip}

\textsuperscript{14} (Paragraph 73 and 153) When someone approves of the purpose for which propaganda is being 
used in a given case, he generally calls it “education” or applies to it some similar 
euphemism. But propaganda is propaganda regardless of the purpose for which it is used. \vspace{\baselineskip}

\textsuperscript{15} (Paragraph 83) We are not expressing approval or disapproval of the Panama invasion. We only 
use it to illustrate a point. \vspace{\baselineskip}

\textsuperscript{16} (Paragraph 95) When the American colonies were under British rule there were fewer and less 
effective legal guarantees of freedom than there were after the American Constitution went into 
effect, yet there was more personal freedom in pre-industrial America, both before and after the 
War of Independence, than there was after the Industrial Revolution took hold in this 
country. We quote from “Violence in America: Historical and Comparative Perspectives,” 
edited by Hugh Davis Graham and Ted Robert Gurr, Chapter 12 by Roger Lane, pages 476-478: 
“The progressive heightening of standards of propriety, and with it the increasing reliance on 
official law enforcement (in l9th century America) ... were common to the whole 
society.... [T]he change in social behavior is so long term and so widespread as to suggest a 
connection with the most fundamental of contemporary social processes; that of industrial 
urbanization itself.... Massachusetts in 1835 had a population of some 660,940, 81 percent rural, 
overwhelmingly preindustrial and native born. It’s citizens were used to considerable personal 
freedom. Whether teamsters, farmers or artisans, they were all accustomed to setting their own 
schedules, and the nature of their work made them physically independent of each 
other.... Individual problems, sins or even crimes, were not generally cause for wider social 
concern....” But the impact of the twin movements to the city and to the factory, both just 
gathering force in 1835, had a progressive effect on personal behavior throughout the 19th 
century and into the 20th. The factory demanded regularity of behavior, a life governed by 
obedience to the rhythms of clock and calendar, the demands of foreman and supervisor. In the 
city or town, the needs of living in closely packed neighborhoods inhibited many actions 
previously unobjectionable. Both blue- and white-collar employees in larger establishments 
were mutually dependent on their fellows; as one man’s work fit into anther’s, so one man’s 
business was no longer his own. The results of the new organization of life and work were 
apparent by 1900, when some 76 percent of the 2,805,346 inhabitants of Massachusetts were 
classified as urbanites. Much violent or irregular behavior which had been tolerable in a casual, 
independent society was no longer acceptable in the more formalized, cooperative atmosphere of 
the later period.... The move to the cities had, in short, produced a more tractable, more 
socialized, more ‘civilized’ generation than its predecessors.” \vspace{\baselineskip}

\textsuperscript{17} (Paragraph 117) Apologists for the system are fond of citing cases in which elections have 
been decided by one or two votes, but such cases are rare. \vspace{\baselineskip}

\textsuperscript{18} (Paragraph 119) “Today, in technologically advanced lands, men live very similar lives in spite 
of geographical, religious, and political differences. The daily lives of a Christian bank clerk in 
Chicago, a Buddhist bank clerk in Tokyo, and a Communist bank clerk in Moscow are far more 
alike than the life of any one of them is like that of any single man who lived a thousand years 
ago. These similarities are the result of a common technology....” L. Sprague de Camp, “The 
Ancient Engineers,” Ballantine edition, page 17. The lives of the three bank clerks are not 
identical. Ideology does have some effect. But all technological societies, in order to survive, 
must evolve along approximately the same trajectory. \vspace{\baselineskip}

\textsuperscript{19} (Paragraph 123) Just think an irresponsible genetic engineer might create a lot of terrorists. \vspace{\baselineskip}

\textsuperscript{20} (Paragraph 124) For a further example of undesirable consequences of medical progress, 
suppose a reliable cure for cancer is discovered. Even if the treatment is too expensive to be 
available to any but the elite, it will greatly reduce their incentive to stop the escape of 
carcinogens into the environment. \vspace{\baselineskip}

\textsuperscript{21} (Paragraph 128) Since many people may find paradoxical the notion that a large number of 
good things can add up to a bad thing, we illustrate with an analogy. Suppose Mr. A is playing 
chess with Mr. B. Mr. C, a Grand Master, is looking over Mr. A’s shoulder. Mr. A of course 
wants to win his game, so if Mr. C points out a good move for him to make, he is doing Mr. A a 
favor. But suppose now that Mr. C tells Mr. A how to make all of his moves. In each 
particular instance he does Mr. A a favor by showing him his best move, but by making all of 
his moves for him he spoils his game, since there is not point in Mr. A’s playing the game at all 
if someone else makes all his moves. The situation of modern man is analogous to that of 
Mr. A. The system makes an individual’s life easier for him in innumerable ways, but in doing 
so it deprives him of control over his own fate. \vspace{\baselineskip}

\textsuperscript{22} (Paragraph 137) Here we are considering only the conflict of values within the 
mainstream. For the sake of simplicity we leave out of the picture “outsider” values like the idea 
that wild nature is more important than human economic welfare. \vspace{\baselineskip}

\textsuperscript{23} (Paragraph 137) Self-interest is not necessarily material self-interest. It can consist in fulfillment 
of some psychological need, for example, by promoting one’s own ideology or religion. \vspace{\baselineskip}

\textsuperscript{24} (Paragraph 139) A qualification: It is in the interest of the system to permit a certain prescribed 
degree of freedom in some areas. For example, economic freedom (with suitable limitations and 
restraints) has proved effective in promoting economic growth. But only planned, 
circumscribed, limited freedom is in the interest of the system. The individual must always be 
kept on a leash, even if the leash is sometimes long (see paragraphs 94, 97). \vspace{\baselineskip}

\textsuperscript{25} (Paragraph 143) We don’t mean to suggest that the efficiency or the potential for survival of a 
society has always been inversely proportional to the amount of pressure or discomfort to which 
the society subjects people. That certainly is not the case. There is good reason to believe that 
many primitive societies subjected people to less pressure than European society did, but European 
society proved far more efficient than any primitive society and always won out in conflicts with 
such societies because of the advantages conferred by technology. \vspace{\baselineskip}

\textsuperscript{26} (Paragraph 147) If you think that more effective law enforcement is unequivocally good 
because it suppresses crime, then remember that crime as defined by the system is not 
necessarily what you would call crime. Today, smoking marijuana is a “crime,” and, in some 
places in the U.S., so is possession of an unregistered handgun. Tomorrow, possession of any 
firearm, registered or not, may be made a crime, and the same thing may happen with 
disapproved methods of child-rearing, such as spanking. In some countries, expression of 
dissident political opinions is a crime, and there is no certainty that this will never happen in the 
U.S., since no constitution or political system lasts forever. If a society needs a large, powerful 
law enforcement establishment, then there is something gravely wrong with that society; it must 
be subjecting people to severe pressures if so many refuse to follow the rules, or follow them 
only because forced. Many societies in the past have gotten by with little or no formal law-
enforcement. \vspace{\baselineskip}

\textsuperscript{27} (Paragraph 151) To be sure, past societies have had means of influencing human behavior, but 
these have been primitive and of low effectiveness compared with the technological means that 
are now being developed. \vspace{\baselineskip} \newpage

\textsuperscript{28} (Paragraph 152) However, some psychologists have publicly expressed opinions indicating 
their contempt for human freedom. And the mathematician Claude Shannon was quoted in 
Omni (August 1987) as saying, “I visualize a time when we will be to robots what dogs are to 
humans, and I’m rooting for the machines.” \vspace{\baselineskip}

\textsuperscript{29} (Paragraph 154) This is no science fiction! After writing paragraph 154 we came across an 
article in Scientific American according to which scientists are actively developing techniques 
for identifying possible future criminals and for treating them by a combination of biological and 
psychological means. Some scientists advocate compulsory application of the treatment, which 
may be available in the near future. (See “Seeking the Criminal Element,” by W. Wayt Gibbs, 
Scientific American, March 1995.) Maybe you think this is okay because the treatment would be 
applied to those who might become violent criminals. But of course it won’t stop there. Next, a 
treatment will be applied to those who might become drunk drivers (they endanger human life 
too), then perhaps to peel who spank their children, then to environmentalists who sabotage 
logging equipment, eventually to anyone whose behavior is inconvenient for the system. \vspace{\baselineskip}

\textsuperscript{30} (Paragraph 184) A further advantage of nature as a counter-ideal to technology is that, in many 
people, nature inspires the kind of reverence that is associated with religion, so that nature could 
perhaps be idealized on a religious basis. It is true that in many societies religion has served as a 
support and justification for the established order, but it is also true that religion has often 
provided a basis for rebellion. Thus it may be useful to introduce a religious element into the 
rebellion against technology, the more so because Western society today has no strong religious 
foundation. Religion, nowadays either is used as cheap and transparent support for narrow, 
short-sighted selfishness (some conservatives use it this way), or even is cynically exploited to 
make easy money (by many evangelists), or has degenerated into crude irrationalism 
(fundamentalist protestant sects, “cults”), or is simply stagnant (Catholicism, main-line 
Protestantism). The nearest thing to a strong, widespread, dynamic religion that the West has 
seen in recent times has been the quasi-religion of leftism, but leftism today is fragmented and 
has no clear, unified, inspiring goal. Thus there is a religious vacuum in our society that could 
perhaps be filled by a religion focused on nature in opposition to technology. But it would be a 
mistake to try to concoct artificially a religion to fill this role. Such an invented religion would 
probably be a failure. Take the “Gaia” religion for example. Do its adherents really believe in it 
or are they just play-acting? If they are just play-acting their religion will be a flop in the end. It 
is probably best not to try to introduce religion into the conflict of nature vs. technology unless 
you really believe in that religion yourself and find that it arouses a deep, strong, genuine 
response in many other people. \vspace{\baselineskip} \newpage

\textsuperscript{31} Assuming that such a final push occurs. Conceivably the industrial system might be eliminated 
in a somewhat gradual or piecemeal fashion (see paragraphs 4, 167 and Note 32). \vspace{\baselineskip}

\textsuperscript{32} (Paragraph 193) It is even conceivable (remotely) that the revolution might consist only of a 
massive change of attitudes toward technology resulting in a relatively gradual and painless 
disintegration of the industrial system. But if this happens we’ll be very lucky. It’s far more 
probably that the transition to a non-technological society will be very difficult and full of 
conflicts and disasters. \vspace{\baselineskip}

\textsuperscript{33} (Paragraph 195) The economic and technological structure of a society are far more important 
than its political structure in determining the way the average man lives (see paragraphs 95, 119 
and Notes 16, 18). \vspace{\baselineskip}

\textsuperscript{34} (Paragraph 215) This statement refers to our particular brand of anarchism. A wide variety of social 
attitudes have been called “anarchist,” and it may be that many who consider themselves anarchists would 
not accept our statement of paragraph 215. It should be noted, by the way, that there is a nonviolent 
anarchist movement whose members probably would not accept FC as anarchist and certainly would not 
approve of FC’s violent methods. \vspace{\baselineskip}

\textsuperscript{35} (Paragraph 219) Many leftists are motivated also by hostility, but the hostility probably results 
in part from a frustrated need for power. \vspace{\baselineskip}

\textsuperscript{36} (Paragraph 229) It is important to understand that we mean someone who sympathizes with 
these movements as they exist today in our society. One who believes that women, 
homosexuals, etc., should have equal rights is not necessary a leftist. The feminist, gay rights, 
etc., movements that exist in our society have the particular ideological tone that characterizes 
leftism, and if one believes, for example, that women should have equal rights it does not 
necessarily follow that one must sympathize with the feminist
\end{document}
